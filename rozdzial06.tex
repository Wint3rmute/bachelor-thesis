\chapter{Podsumowanie}

W ramach projektu stworzony został system inspekcji obszarów, wykorzystujący autonomiczne drony.
W trakcie rozwoju system był regularnie testowany, zarówno na poziomie poszczególnych komponentów
jak i na poziomie całego systemu. Dzięki temu, w trakcie testów w terenie wszystkie
komponenty systemu bezkonfliktowo współgrały
ze sobą. Z punktu widzenia infrastruktury sieciowej,
testy w terenie nie różniły się niczym od testów wykonywanych na symulatorze --
dane telemetryczne odbierane z drona były dokładnie takie~same.

Osiągniętą dokładność rozpoznawania ludzi na zdjęciach wykonanych w trakcie lotu
można uznać za zadowalającą, zważywszy na fakt że praca nie skupiała się na
algorytmach rozpoznawania obrazu. Wykorzystanie otwartych zbiorów danych, zawierających
oznakowane zdjęcia wykonane z samolotów i dronów (przykładowo \textit{VisDrone Dataset} \cite{visdrone}),
może poprawnić dokładność rozpoznawania obiektów na zdjęciach.

Jak wspomniano w podrozdziale \ref{early_tests}, automatyczne aktualizowanie infrastruktury
internetowej systemu pozwoliło na szybsze i bezpieczniejsze wprowadzanie poprawek w systemie.
Dzięki temu udało się uniknąć potencjalnych błędów przy wdrażaniu nowych funkcjonalności.

Zastosowane w pracy rozwiązania kwalifikują się do metodyki \textit{DevOps}, opierającej
się na zacieśnieniu więzów pomiędzy programistami i administratorami systemu.
Automatyzacja procesów związanych z testami i wdrożeniami oraz powiązanie ich
z repozytorium projektowym zwiększa u programistów świadomość tego, jak ważny jest proces wdrażania.
W przypadku systemu wykorzystującego realne, fizyczne i drogie komponenty, pozwala to 
przyspieszyć ,,dojrzewanie'' systemu -- czas, po którym programiści mogą być pewni stabilności
działania produktu, nad którym pracują.

W początkowej fazie rozwoju projektu, dodatkowy wysiłek związany z budową infrastruktury
odpowiedzialnej za wdrożenia może wydawać się zupełnie zbędny. Nie jest to jednak 
błąd, jak w przypadku przedwczesnej optymalizacji. Wczesne zdefiniowanie ram projektu
pozwala na utrzymanie rozwoju kodu w ryzach. Jest niezbędne w celu utrzymania
stałego kursu ku wyznaczonym w projekcie celom.

% nk zdanie z agregacją entropii
