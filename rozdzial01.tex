\chapter{Wstęp}


\section{Geneza pracy}
% kilka akapitów wprowadzających,
% z czego powinno wynikać, że zrobienie
% takiego systemu jest potrzebne i ma sens.

Autonomiczne lotnictwo to dynamicznie rozwijający się sektor branży lotniczej.
Technologie pozwalające na wykorzystanie autonomicznych dronów i samolotów
w nowych projektach biznesowych są powszechnie dostępne. Istnieją
zarówno systemy zamknięte, w pełni komercyjne, jak i projekty otwartoźródłowe,
pozwalające na zapoznanie się z kodem źródłowym oprogramowania sterującego statkami
powietrznymi i interakcję z aktywną społecznością pasjonatów, wspólnie rozwijającą
projekt. 

W świecie biznesu powstają coraz to nowe rozwiązania, wykorzystujące autonomiczne
maszyny do świadczenia usług - od razu nasuwającym się rozwiązaniem
jest autonomiczne dostarczanie paczek \cite{prime_air}, ale istnieją też znacznie
bardziej ambitne projekty: dostarczanie defibrylatora wprost do domu osoby, u której
wystąpiło zatrzymanie akcji serca \cite{10.1001/jama.2017.3957} lub generowanie
trójwymiarowych map terenu, wykorzystując zdjęcia wykonanych z dronów
\cite{uav_photogrametry}. Biznes jest stosunkowo młody, więc branża jest otwarta
na innowatorów -- firmy takie jak Boeing i Lockheed Martin sponsorują międzynarodowe
konkursy przeznaczone dla młodych konstruktorów \cite{sae_2018}. 

Zainteresowani autonomicznym lotnictwem inwestorzy nie ograniczają się do
prywatnych firm. Rząd australijskiego stanu Queensland współorganizuje
\textit{UAV Challenge} -- zawody skupione wokół rozwijania systemów wspierających 
służby medyczne \cite{uav_sponsors}.

Wykorzystanie otwartych technologii, skupionych wokół awiacji autonomicznej
i połączenie ich z nowoczesnymi praktykami wdrażania oprogramowania to temat atrakcyjny
zarówno z perspektywy inżynierii oprogramowania, jak i z perspektywy biznesowej. 
% zapytać o to
Koło naukowe Politechniki Wrocławskiej -- \textit{Akademicki Klub Lotniczy} (AKL) zajmuje
się budowaniem i rozwijaniem autonomicznych dronów i samolotów \cite{akl_home_page}.
Omawiany w pracy system powstał w ramach realizacji projektu
z Wrocławskiego Funduszu Aktywności Studenckiej\cite{fast_webpage}.

% Szczególnie interesujące są zagadnienia integracji komponentów systemu,
% oraz testowanie - które w przypadku systemu angażującego rzeczywiste
% maszyny nie może ograniczyć się jedynie do standardowych testów jednostkowych.

\newpage
\section{Cel pracy}
% koniecznie sformułowania:
% - Celem ogólnym pracy jest…?,
% - Celami szczegółowymi pracy są…?

Celem pracy jest stworzenie prototypu systemu inspekcji terenów, wykorzystującego
autonomiczne drony. System ma integrować się z już istniejącym oprogramowaniem
sterującym autonomicznymi maszynami, oraz wykorzystywać napisaną na potrzeby pracy
infrastrukturę informatyczną, pozwalającą na planowanie tras lotów, obsługę telemetrii
i rozpoznawanie obiektów na zdjęciach wykonanych w czasie lotu, za pomocą algorytmów
sztucznej inteligencji. Finalnie, system ma generować raport podsumowujący każdy lot,
w którym zawarta będzie trasa pokonana przez drona i zdjęcia wykonane w trakcie lotu,
wraz z rozpoznanymi na nich obiektami. 

Architektura systemu musi pozwalać na zautomatyzowanie procesu wdrażania
systemu, oraz zautomatyzowanie wdrażania nowych funkcjonalności - każde
z wdrożeń musi być poprzedzone testami integracyjnymi na poziomie całego systemu. 

Prototyp ma być w pełni testowalny, zarówno na poziomie pojedynczych
elementów systemu jak i na poziomie integracji całego projektu - testy muszą
angażować wszystkie komponenty systemu, uruchomione wewnątrz w pełni
zautomatyzowanego środowiska testowego.

\section{Zakres pracy}

Zakres pracy obejmuje elementy projektu związane z
inżynierią i architekturą oprogramowania - proces projektowania struktury systemu,
wybór technologii, zaprojektowanie punktów stykowych w systemie, automatyzacja
procesu wdrażania systemu i nowych funkcjonalności.

Praca opisuje także sposób testowania systemu - od weryfikacji poprawności działania
poszczególnych komponentów, po pełne automatyczne testy integracyjne, wykorzystujące 
wszystkie komponenty systemu wraz ze zintegrowanym symulatorem drona. 
