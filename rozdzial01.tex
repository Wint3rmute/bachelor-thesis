\chapter{Wstęp} \label{chapter_intro}


\section{Geneza pracy} \label{intro_genesis}
% kilka akapitów wprowadzających,
%~z czego powinno wynikać, że zrobienie
% takiego systemu jest potrzebne ~i$3ma sens.

Sektor gospodarki związany~z inżynierią bezzałogowych maszyn lotniczych znajduje się~w
stanie dynamicznego rozwoju~i generuje coraz większe przychody.
Tym samym przyciąga inwestorów skłonnych zainwestować~w innowacyjne pomysły zarówno 
dużych przedsiębiorstw, jak~i młodych konstruktorów. Międzynarodowe zawody sponsorowane
przez podmioty prywatne~i organizacje rządowe umożliwiają pasjonatom wdrożenie~w życie
ich wizji. Przykład stanowi \textit{UAV Challenge} -- zawody skoncentrowane na
autonomicznych systemach wsparcia służb medycznych, współorganizowane przez rząd
australijskiego stanu Queensland, które przyciągają corocznie najlepsze zespoły~z całego
świata\cite{uav_sponsors}. Potencjał inwestycyjny~i naukowy, którym dysponuje ta branża,
sprzyja powstawaniu różnorodnych rozwiązań.~W konsekwencji bezpośrednio niezwiązane~z lotnictwem sektory korzystają~z dorobku autonomicznej awiacji.
Tytułem przykładu: generowanie trójwymiarowych map terenu, przy użyciu zdjęć wykonanych~z dronów znajduje
zastostowanie~w dziedzinach takich jak górnictwo, geodezja czy militaria
\cite{uav_photogrametry}, natomiast dostarczanie towarów indywidualnym klientom~z pomocą
dronów jest~w stanie zrewolucjonizować przemysł transportowy \cite{prime_air}.

System realizowany~w ramach pracy mógłby potencjalnie służyć jako wsparcie dla służb
medycznych, wspomagając poszukiwanie osób zaginionych. Jest jednak to jedynie pojedynczy 
przykład z szerokiego spektrum potencjalnych zastosowań. Algorytmy sztucznej inteligencji,
zastosowane do analizy zdjęć wykonanych~w czasie lotów, mogą zostać wytrenowane
na nowych danych, aby rozpoznawać dowolne obiekty (samochody, budynki, drzewa). 
Dzięki temu, system może zostać szybko przystosowany do nowych celów.

Budowa systemu wykorzystującego fizyczne komponenty, takie jak autonomiczne drony,
wymaga wprowadzenia dodatkowych zabezpieczeń, pozwalających na przetestowanie stabilności
systemu, zanim zostanie uruchomiony w terenie. Wymaganie podyktowane jest wysokim kosztem
dronów oraz względami bezpieczeństwa -- utrata kontroli nad maszyną latającą, gdy wokół znajdują
się ludzie, stanowi duże zagrożenie. Nowoczesne praktyki inżynierii oprogramowania, takie jak 
automatyczne testy w potoku \textit{CI/CD} oraz zastosowanie symulatorów i atrap
w miejsce rzeczywistych, fizycznych obiektów pozwalają na budowę takich zabezpieczeń.

Przedstawiony~w pracy prototyp systemu inspekcji obszarów powstał~w ramach działalności
Akademickiego Klubu Lotniczego -- koła naukowego Politechniki Wrocławskiej, zajmującego
się rozwijaniem technologii związanych~z autonomicznymi dronami~i samolotami
\cite{akl_home_page}. Projekt został sfinansowany przez Wrocławski Fundusz
Aktywności Studenckiej \cite{fast_webpage}.

% Szczególnie interesujące są zagadnienia integracji komponentów systemu,
% oraz testowanie - które~w przypadku systemu angażującego rzeczywiste
% maszyny nie może ograniczyć się jedynie do standardowych testów jednostkowych.

% \newpage
\section{Cel pracy} \label{intro_objective}
% koniecznie sformułowania:
% - Celem ogólnym pracy jest…?,
% - Celami szczegółowymi pracy są…?

Celem pracy jest zaprojektowanie architektury i budowa prototypu systemu inspekcji obszarów,
wykorzystującego autonomiczne drony. %realizowanego w ramach działalności koła naukowego Akademicki Klub Lotniczy.
% TODO: FIX SYSTEM MA WYKORZYSTYWAĆ
System ma wykorzystywać napisaną~w ramach pracy
infrastrukturę informatyczną, pozwalającą na planowanie tras lotów, obsługę telemetrii~i
rozpoznawanie obiektów na zdjęciach wykonanych~w czasie lotu, za pomocą algorytmów
sztucznej inteligencji. 

Architektura systemu (wykorzystane technologie~i struktura podprojektów składających
się na system) musi pozwalać na zautomatyzowanie procesu testowania i wdrażania systemu
 -- każde~z wdrożeń musi być poprzedzone testami systemowymi.
Aby przeprowadzić pełne testy systemowe, prototyp systemu będzie uruchamiany~w
środowisku testowym, zawierającym zintegrowany symulator drona/samolotu.

Finalnie, system ma pozwolić użytkownikowi na zaplanowanie trasy przelotu drona
oraz wyznaczenie harmonogramu, według którego mają odbywać się loty.~W trakcie 
lotu, system ma~w czasie rzeczywistym dostarczać użytkownikowi dane telemetryczne~o
obecnej pozycji maszyny~i pozwalać na podgląd zdjęć wykonywanych~w czasie lotu.
Po wylądowaniu, system wygeneruje dla użytkownika podsumowanie lotu, zawierające
trasę pokonaną przez drona, wraz~z wykonanymi zdjęciami, na których oznaczone zostaną
rozpoznane przez algorytmy sztucznej inteligencji obiekty.

\section{Zakres~i struktura pracy} \label{intro_scope}

Zakres pracy obejmuje elementy projektu związane~z inżynierią~i architekturą
oprogramowania oraz proces budowy i testowania prototypu systemu. Poszczególne
rozdziały pracy skupione są wokół różnych zadań, realizowanych w ramach projektu: 

\begin{itemize}
    \item zdefiniowanie wymagań funkcjonalnych systemu -- rozdział \ref{chapter_functional_requirements},
    \item projekt architektury systemu -- rozdział \ref{chapter_architecture},
    \item opis zastosowanej metodyki wdrażania i jej wpływ na rozwój projektu -- rozdział \ref{chapter_deployment}. 
    \item zaprojektowanie i przeprowadzenie testów -- rozdział \ref{chapter_tests}.
\end{itemize}
