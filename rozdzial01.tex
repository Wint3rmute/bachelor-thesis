\chapter{Wstęp}



\section{Geneza pracy}
% kilka akapitów wprowadzających,
% z czego powinno wynikać, że zrobienie
% takiego systemu jest potrzebne i ma sens.

Lotnictwo autonomiczne to prężnie rozwijający się segment branży lotniczej.
Technologie pozwalające na wykorzystanie autonomicznych dronów i samolotów
w nowych projektach biznesowych są dostępne na wyciągnięcie ręki - istnieją
zarówno systemy zamknięte, w pełni komercyjne, jak i projekty zupełnie otwarte,
pozwalające na zapoznanie się z kodem źródłowym oprogramowania sterującego statkami
powietrznymi. 

W świecie biznesu powstają coraz to nowe rozwiązania, wykorzystujące autonomiczne
maszyny do świadczenia różnorakich usług - od razu nasuwającym się rozwiązaniem
jest autonomiczne dostarczanie paczek \cite{prime_air}. Warto wspomnieć, że branża
jest bardzo otwarta na innowatorów - firmy takie jak Boeing i Lockheed Martin
sponsorują międzynarodowe konkursy dla młodych konstruktorów \cite{sae_2018}. 

Zainteresowani autonomicznym lotnictwem inwestorzy nie ograniczają się do
prywatnych firm. Rząd australijskiego stanu Queensland współorganizuje
UAV Challenge - zawody skupione wokół rozwijania systemów wspierających 
służby medyczne \cite{uav_sponsors}.


Wykorzystanie otwartych technologii skupionych wokół awiacji autonomicznej
i połączenie ich z nowoczesnymi praktykami wdrażania oprogramowania to temat
atrakcyjny zarówno z perspektywy inżynierii oprogramowania jak i z perspektywy biznesowej. 

Szczególnie interesujące są zagadnienia integracji komponentów systemu,
oraz testowanie - które w przypadku systemu angażującego rzeczywiste
maszyny nie może ograniczyć się jedynie do standardowych testów jednostkowych.


\section{Cel pracy}
% koniecznie sformułowania:
% - Celem ogólnym pracy jest…?,
% - Celami szczegółowymi pracy są…?

Celem pracy jest stworzenie prototypu systemu monitorującego, wykorzystującego
autonomiczne drony. System ma wykorzystywać już istniejące oprogramowanie sterujące
autonomicznymi maszynami, wykorzystywać napisaną na potrzeby pracy infrastrukturę
służącą do planowania tras lotów, przechwytywanie i wyświetlanie telemetrii oraz 
rozpoznawanie obiektów na zdjęciach wykonanych w czasie lotu za pomocą sztucznej
inteligencji.

Prototyp ma być w pełni testowalny - testy muszą angażować wszystkie
komponenty systemu, uruchomione wewnątrz w pełni zautomatyzowanego środowiska
testowego.

\section{Zakres pracy}

Zakres pracy obejmuje elementy projektu związane z
inżynierią oprogramowania - proces projektowania architektury systemu,
wybór technologii, zaprojektowanie punktów stykowych w systemie. Praca opisuje
też sposób testowania systemu - od weryfikacji poprawności działania poszczególnych
komponentów, po pełne automatyczne testy integracyjne, wykorzystujące wszystkie
komponenty systemu oraz zintegrowany symulator drona. 
