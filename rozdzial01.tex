\chapter{Wstęp} \label{chapter_intro}


\section{Geneza pracy} \label{intro_genesis}
% kilka akapitów wprowadzających,
%~z czego powinno wynikać, że zrobienie
% takiego systemu jest potrzebne ~i$3ma sens.

Sektor gospodarki związany~z inżynierią bezzałogowych maszyn lotniczych znajduje się~w
stanie dynamicznego rozwoju~i generuje coraz większe przychody.
Tym samym przyciąga inwestorów skłonnych zainwestować~w innowacyjne pomysły zarówno 
dużych przedsiębiorstw, jak~i młodych konstruktorów. Międzynarodowe zawody sponsorowane
przez podmioty prywatne~i organizacje rządowe umożliwiają pasjonatom wdrożenie~w życie
ich wizji. Przykład stanowi \textit{UAV Challenge} -- zawody skoncentrowane na
autonomicznych systemach wsparcia służb medycznych, współorganizowane przez rząd
australijskiego stanu Queensland, które przyciągają corocznie najlepsze zespoły~z całego
świata\cite{uav_sponsors}. Potencjał inwestycyjny~i naukowy, którym dysponuje ta branża,
sprzyja powstawaniu różnorodnych rozwiązań.~W konsekwencji bezpośrednio niezwiązane~z lotnictwem sektory korzystają~z dorobku autonomicznej awiacji.
Tytułem przykładu: generowanie trójwymiarowych map terenu, przy użyciu zdjęć wykonanych~z dronów znajduje
zastostowanie~w dziedzinach takich jak górnictwo, geodezja czy militaria
\cite{uav_photogrametry}, natomiast dostarczanie towarów indywidualnym klientom~z pomocą
dronów jest~w stanie zrewolucjonizować przemysł transportowy \cite{prime_air}.

% Przeprowadzenie autonomicznej misji bezzałogową maszyną możliwe jest dzięki osiągnięciom
% inżynierii oprogramowania. Entuzjaści programowania regularnie udoskonalają~i rozbudowują
% projekty open-source. Powszechna dostępność~i możliwość korzystania ze sprawdzonych
%~i stabilnych rozwiązań zdecydowanie usprawnia proces realizowania
% kreatywnych~i innowacyjnych projektów. 

System realizowany~w ramach pracy mógłby potencjalnie służyć jako wsparcie dla służb
medycznych, wspomagając poszukiwanie osób zaginionych. Jest jednak to jedynie pojedynczy 
przykład z szerokiego spektrum potencjalnych zastosowań. Algorytmy sztucznej inteligencji,
zastosowane do analizy zdjęć wykonanych~w czasie lotów, mogą zostać wytrenowane
na nowych danych, aby rozpoznawać dowolne obiekty (samochody, budynki, drzewa). 
Dzięki temu, system może zostać szybko przystosowany do nowych wymagań. 

Przedstawiony~w pracy prototyp systemu inspekcji obszarów powstał~w ramach działalności
Akademickiego Klubu Lotniczego -- koła naukowego Politechniki Wrocławskiej, zajmującego
się rozwijaniem technologii związanych~z autonomicznymi dronami~i samolotami
\cite{akl_home_page}. Projekt został sfinansowany przez Wrocławski Fundusz
Aktywności Studenckiej \cite{fast_webpage}.

  

% Szczególnie interesujące są zagadnienia integracji komponentów systemu,
% oraz testowanie - które~w przypadku systemu angażującego rzeczywiste
% maszyny nie może ograniczyć się jedynie do standardowych testów jednostkowych.

\newpage
\section{Cel pracy} \label{intro_objective}
% koniecznie sformułowania:
% - Celem ogólnym pracy jest…?,
% - Celami szczegółowymi pracy są…?

Celem pracy jest wykorzystanie nowoczesnych praktyk~z zakresu inżynierii 
oprogramowania~w celu stworzenia prototypu systemu inspekcji obszarów,
% TODO: FIX SYSTEM MA WYKORZYSTYWAĆ
wykorzystującego autonomiczne drony. System ma wykorzystywać napisaną~w ramach pracy
infrastrukturę informatyczną, pozwalającą na planowanie tras lotów, obsługę telemetrii~i
rozpoznawanie obiektów na zdjęciach wykonanych~w czasie lotu, za pomocą algorytmów
sztucznej inteligencji. 

Architektura systemu (wykorzystane technologie~i struktura podprojektów składających
się na system) musi pozwalać na zautomatyzowanie procesu wdrażania
systemu, oraz zautomatyzowanie wdrażania nowych funkcjonalności -- każde~z wdrożeń musi
być poprzedzone testami integracyjnymi na poziomie całego systemu.
Aby przeprowadzić pełne testy integracyjne, prototyp systemu będzie uruchamiany~w
środowisku testowym, zawierającym zintegrowany symulator drona/samolotu.

Finalnie, system ma pozwolić użytkownikowi na zaplanowanie trasy przelotu drona
oraz wyznaczenie harmonogramu, według którego mają odbywać się loty.~W trakcie 
lotu, system ma~w czasie rzeczywistym dostarczać użytkownikowi dane telemetryczne~o
obecnej pozycji maszyny~i pozwalać na podgląd zdjęć wykonywanych~w czasie lotu.
Po wylądowaniu, system wygeneruje dla użytkownika podsumowanie lotu, zawierające
trasę pokonaną przez drona, wraz~z wykonanymi zdjęciami, na których oznaczone zostaną
rozpoznane przez algorytmy sztucznej inteligencji obiekty.

\section{Zakres~i struktura pracy} \label{intro_scope}

Zakres pracy obejmuje elementy projektu związane~z inżynierią~i architekturą
oprogramowania -- proces projektowania struktury systemu,
wybór technologii, zaprojektowanie punktów stykowych~w systemie, automatyzacja
procesu wdrażania~i testowania systemu.

W rozdziale \ref{chapter_functional_requirements} opisane są wymagania funkcjonalne
każdego z komponentów systemu. Następnie, rozdział \ref{chapter_architecture} opisuje
proces wyboru technologii i frameworków zastosowanych do budowy systemu. 
Rozwiązania wykorzystane w celu automatyzacji budowania i wdrażania systemu opisane
są w rozdziale \ref{chapter_deployment}. Testy systemu (zarówno integracyjne jak
i w terenie) opisuje rozdział~\ref{chapter_tests}.
