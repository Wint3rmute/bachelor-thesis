\chapter{Wstęp}


\section{Geneza pracy} \label{intro_genesis}
% kilka akapitów wprowadzających,
% z czego powinno wynikać, że zrobienie
% takiego systemu jest potrzebne i ma sens.

Jako członek Akademickiego Klubu Lotniczego - koła naukowego Politechniki Wrocławskiej,
zajmuję się oprogramowywaniem autonomicznych dronów i samolotów. W ramach realizacji
projektu z Wrocławskiego Funduszu Aktywności Studenckiej powstał, przedstawiony w
tej pracy, prototyp systemu inspekcji obszarów.

Sektor gospodarki związany z inżynierią bezzałogowych maszyn lotniczych znajduje się
w stanie dynamicznego rozwoju i generuje coraz większe przychody.
Tym samym przyciąga inwestorów skłonnych zainwestować w innowacyjne pomysły zarówno 
dużych przedsiębiorstw, jak i młodych konstruktorów. Międzynarodowe zawody sponsorowane
przez podmioty prywatne i organizacje rządowe umożliwiają pasjonatom wdrożenie w życie
ich wizji. Przykład stanowi \textit{UAV Challenge} - zawody skoncentrowane na
autonomicznych systemach wsparcia służb medycznych, współorganizowane przez rząd
australijskiego stanu Queensland, które przyciągają corocznie najlepsze zespoły z całego
świata\cite{uav_sponsors}. Potencjał inwestycyjny i naukowy, którym dysponuje ta branża,
sprzyja powstawaniu różnorodnych rozwiązań. W konsekwencji bezpośrednio niezwiązane z
lotnictwem sektory korzystają z dorobku autonomicznej awiacji. Tytułem przykładu: 
generowanie trójwymiarowych map terenu, przy użyciu zdjęć wykonanych z dronów znajduje
zastostowanie w dziedzinach takich jak górnictwo, geodezja czy militaria
\cite{uav_photogrametry}, natomiast dostarczanie towarów indywidualnym klientom z pomocą
dronów jest w stanie zrewolucjonizować przemysł transportowy.

Przeprowadzenie autonomicznej misji bezzałogową maszyną możliwe jest dzięki osiągnięciom
inżynierii oprogramowania. Entuzjaści programowania regularnie udoskonalają i rozbudowują
projekty open-source. Powszechna dostępność i możliwość korzystania ze sprawdzonych
i stabilnych rozwiązań zdecydowanie usprawnia proces realizowania
kreatywnych i innowacyjnych projektów. 


  

% Autonomiczne lotnictwo to dynamicznie rozwijający się sektor branży lotniczej.
% Technologie pozwalające na wykorzystanie autonomicznych dronów i samolotów
% w nowych projektach biznesowych są powszechnie dostępne. Istnieją
% zarówno systemy zamknięte, w pełni komercyjne, jak i projekty otwartoźródłowe,
% pozwalające na zapoznanie się z kodem źródłowym oprogramowania sterującego statkami
% powietrznymi i interakcję z aktywną społecznością pasjonatów, wspólnie rozwijającą
% projekt. 

% W świecie biznesu powstają coraz to nowe rozwiązania, wykorzystujące autonomiczne
% maszyny do świadczenia usług - od razu nasuwającym się rozwiązaniem
% jest autonomiczne dostarczanie paczek \cite{prime_air}, ale istnieją też znacznie
% bardziej ambitne projekty: dostarczanie defibrylatora wprost do domu osoby, u której
% wystąpiło zatrzymanie akcji serca \cite{10.1001/jama.2017.3957} lub generowanie
% trójwymiarowych map terenu, wykorzystując zdjęcia wykonanych z dronów
% \cite{uav_photogrametry}. Biznes jest stosunkowo młody, więc branża jest otwarta
% na innowatorów -- firmy takie jak Boeing i Lockheed Martin sponsorują międzynarodowe
% konkursy przeznaczone dla młodych konstruktorów \cite{sae_2018}. 

% Zainteresowani autonomicznym lotnictwem inwestorzy nie ograniczają się do
% prywatnych firm. Rząd australijskiego stanu Queensland współorganizuje
% \textit{UAV Challenge} -- zawody skupione wokół rozwijania systemów wspierających 
% służby medyczne \cite{uav_sponsors}.

% Wykorzystanie otwartych technologii, skupionych wokół awiacji autonomicznej
% i połączenie ich z nowoczesnymi praktykami wdrażania oprogramowania to temat atrakcyjny
% zarówno z perspektywy inżynierii oprogramowania, jak i z perspektywy biznesowej. 
% % zapytać o to
% Koło naukowe Politechniki Wrocławskiej -- \textit{Akademicki Klub Lotniczy} (AKL) zajmuje
% się budowaniem i rozwijaniem autonomicznych dronów i samolotów \cite{akl_home_page}.
% Omawiany w pracy system powstał w ramach realizacji projektu
% z Wrocławskiego Funduszu Aktywności Studenckiej\cite{fast_webpage}.




% Szczególnie interesujące są zagadnienia integracji komponentów systemu,
% oraz testowanie - które w przypadku systemu angażującego rzeczywiste
% maszyny nie może ograniczyć się jedynie do standardowych testów jednostkowych.

\newpage
\section{Cel pracy} \label{intro_objective}
% koniecznie sformułowania:
% - Celem ogólnym pracy jest…?,
% - Celami szczegółowymi pracy są…?

Celem pracy jest stworzenie prototypu systemu inspekcji terenów, wykorzystującego
autonomiczne drony. System ma integrować się z już istniejącym oprogramowaniem
sterującym autonomicznymi maszynami, oraz wykorzystywać napisaną na potrzeby pracy
infrastrukturę informatyczną, pozwalającą na planowanie tras lotów, obsługę telemetrii
i rozpoznawanie obiektów na zdjęciach wykonanych w czasie lotu, za pomocą algorytmów
sztucznej inteligencji. Finalnie, system ma generować raport podsumowujący każdy lot,
w którym zawarta będzie trasa pokonana przez drona i zdjęcia wykonane w trakcie lotu,
wraz z rozpoznanymi na nich obiektami. 

Architektura systemu musi pozwalać na zautomatyzowanie procesu wdrażania
systemu, oraz zautomatyzowanie wdrażania nowych funkcjonalności - każde
z wdrożeń musi być poprzedzone testami integracyjnymi na poziomie całego systemu. 

Prototyp ma być w pełni testowalny, zarówno na poziomie pojedynczych
elementów systemu jak i na poziomie integracji całego projektu - testy muszą
angażować wszystkie komponenty systemu, uruchomione wewnątrz w pełni
zautomatyzowanego środowiska testowego.

\section{Zakres pracy} \label{intro_scope}

Zakres pracy obejmuje elementy projektu związane z
inżynierią i architekturą oprogramowania - proces projektowania struktury systemu,
wybór technologii, zaprojektowanie punktów stykowych w systemie, automatyzacja
procesu wdrażania systemu i nowych funkcjonalności.

Praca opisuje także sposób testowania systemu - od weryfikacji poprawności działania
poszczególnych komponentów, po pełne automatyczne testy integracyjne, wykorzystujące 
wszystkie komponenty systemu wraz ze zintegrowanym symulatorem drona. 
