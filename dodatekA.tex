\chapter{Tytuł dodatku}
Zasady przyznawania stopnia naukowego doktora i doktora habilitowanego w Polsce określa ustawa z dnia 14 marca 2003 r. o stopniach naukowych i~tytule naukowym oraz o stopniach i~tytule w zakresie sztuki (Dz.U. nr 65 z 2003 r., poz. 595 (Dz. U. z 2003 r. Nr 65, poz. 595). Poprzednie polskie uregulowania nie wymagały bezwzględnie posiadania przez kandydata tytułu zawodowego magistra lub równorzędnego (choć zasada ta zazwyczaj była przestrzegana) i zdarzały się nadzwyczajne przypadki nadawania stopnia naukowego doktora osobom bez studiów wyższych, np. słynnemu matematykowi lwowskiemu – późniejszemu profesorowi Stefanowi Banachowi. 

W innych krajach również zazwyczaj do przyznania stopnia naukowego doktora potrzebny jest dyplom ukończenia uczelni wyższej, ale nie wszędzie.

