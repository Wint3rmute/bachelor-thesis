%%%%%%%%%%%%%%%%%%%%%%%%%%%%%%%%%%%%%%%%%%%%%%%%%%%%%%%%%%%%%%%%%%%%%%%%%
%  Zawartość: Główny plik szablonu pracy dyplomowej (magisterskiej/inżynierskiej).
%  Opracował: Tomasz Kubik <tomasz.kubik@pwr.edu.pl>
%  Data: 1 marca 2020
%  Wersja: 0.4
%%%%%%%%%%%%%%%%%%%%%%%%%%%%%%%%%%%%%%%%%%%%%%%%%%%%%%%%%%%%%%%%%%%%%%%%%

\documentclass[a4paper,onecolumn,oneside,12pt,extrafontsizes]{memoir}
% W celu przygotowania wydruku  do archiwum można:
% a) przygotować pdf, w którym dwie strony zostaną wstawione na jedną fizyczną stronę i taki dokument wydrukować dwustronnie (podejście zalecane)
%
%   Taki dokument można przygotować poprzez
%   - wydruk z Adobe Acrobat Reader z opcją "Wiele" - sekcja "Rozmiar i obsługa stron"
%   - wykorzystanie narzędzi psutils
%
%      Windows (zakładając, że w dystrybucji MiKTeX jest pakiet miktex-psutils-bin-x64-2.9):
%        "c:\Program Files\MiKTeX 2.9\miktex\bin\x64\pdf2ps.exe" Dyplom.pdf Dyplom.ps
%        "c:\Program Files\MiKTeX 2.9\miktex\bin\x64\psnup.exe" -2 Dyplom.ps Dyplom2.ps
%        "c:\Program Files\MiKTeX 2.9\miktex\bin\x64\ps2pdf.exe" Dyplom2.ps Dyplom2.pdf
%        Del Dyplom2.ps Dyplom.ps
%
%     Linux:
%        pdf2ps Dyplom.pdf - | psnup -2 | ps2pdf - Dyplom2.pdf
%
%
% b) przekomplilować dokument zmniejszając czcionkę (podejście niezalecane, bo zmienia formatowanie dokumentu)
%
%    Do tego wystarczy posłużyć się dwoma poniższymi komendami (zamiast documentclass z pierwszej linijki):
%   \documentclass[a4paper,onecolumn,twoside,10pt]{memoir} 
%   \renewcommand{\normalsize}{\fontsize{8pt}{10pt}\selectfont}

%\usepackage[cp1250]{inputenc} % jeśli kodowanie edytowanych plików to cp1250 
\usepackage[utf8]{inputenc} % jeśli kodowanie edytowanych plików to UTF8
\usepackage[T1]{fontenc}
\usepackage[polish]{babel}
%\DisemulatePackage{setspace}
\usepackage{setspace}
\usepackage{tabularx}
\usepackage{color,calc}
%\usepackage{soul} % pakiet z komendami do podkreślania tekstu

\usepackage{ebgaramond} % pakiet z czcionkami garamond, potrzebny tylko do strony tytułowej, musi wystąpić przed pakietem tgtermes

%% Aby uzyskać polskie literki w pdfie (a nie zlepki) korzystamy z pakietu czcionek tgterms. 
%% W pakiecie tym są zdefiniowane klony czcionek Times o kształtach: normalny, pogrubiony, italic, italic pogrubiony.
%% W pakiecie tym brakuje czcionki o kształcie: slanted (podobny do italic). 
%% Jeśli w dokumencie gdzieś zostanie zastosowana czcionka slanted (np. po użyciu komendy \textsl{}), to
%% latex dokona podstawienia na czcionkę standardową i zgłosi to w ostrzeżeniu (warningu).
%% Ponadto tgtermes to czcionka do tekstu. Wszelkie matematyczne wzory będą sformatowane domyślną czcionką do wzorów.
%% Jeśli wzory mają być sformatowane z wykorzystaniem innych czcionek, trzeba to jawnie zadeklarować.

%% BĄCZKOWE BIBLIOTEKI

\usepackage{tikz}
\usetikzlibrary{er,positioning}

\usepackage[edges]{forest}
\forestset{
  direction switch/.style={
    for tree={edge+=thick, font=\sffamily},
    where level>=1{folder, grow'=0}{for children=forked edge},
    where level=3{}{draw},
  },
}

\usepackage{float}
\usepackage{multicol}
% \usepackage{lscape}
% we want ER + above/below + left/right

%% END OF BĄCZKOWE BIBLIOTEKI

%% Po zainstalowaniu pakietu tgtermes może będzie trzeba zauktualizować informacje 
%% o dostępnych fontach oraz mapy. Można to zrobić z konsoli (jako administrator)
%% initexmf --admin --update-fndb
%% initexmf --admin --mkmaps

\usepackage{tgtermes}   
\renewcommand*\ttdefault{txtt}

% We wcześniejszej wersji szablonu korzystano z innych czcionek. Dla celów historycznych pozostawiono je w komentarzu
%\usepackage{mathptmx} % pakiet będący następcą pakietów times and mathptm, niestety polskie literki są zlepkami
%\usepackage{newtxtext,newtxmath} % pakiety dostarczające Times dla tekstów i wzorów matematycznych,  
%                                  rozwiązuje problemy występujące w mathptmx, ale wymaga zainstalowania
%                                  dodatkowych pakietów oraz uruchomienia updmap (konsola administratora)
%                                  niestety polskie literki są zlepkami
%\usepackage{newtxmath,tgtermes} % można też połączyć czcionki do tekstu i czcionki do wzorów

\usepackage{listings} % pakiet do prezentacji kodu. 
% Wcześniej był problem z polskimi znakami w otoczeniu lstlisting, stąd poniższe rozwiązanie: 
\lstset{literate=%-
{ą}{{\k{a}}}1 {ć}{{\'c}}1 {ę}{{\k{e}}}1 {ł}{{\l{}}}1 {ń}{{\'n}}1 {ó}{{\'o}}1 {ś}{{\'s}}1 {ż}{{\.z}}1 {ź}{{\'z}}1 {Ą}{{\k{A}}}1 {Ć}{{\'C}}1 {Ę}{{\k{E}}}1 {Ł}{{\L{}}}1 {Ń}{{\'N}}1 {Ó}{{\'O}}1 {Ś}{{\'S}}1 {Ż}{{\.Z}}1 {Ź}{{\'Z}}1 
    {Ö}{{\"O}}1
    {Ä}{{\"A}}1
    {Ü}{{\"U}}1
    {ß}{{\ss}}1
    {ü}{{\"u}}1
    {ä}{{\"a}}1
    {ö}{{\"o}}1
    {~}{{\textasciitilde}}1
		{—}{{{\textemdash} }}1
}%{\ \ }{{\ }}1}


% KOD DODANY PRZEZ BĄCZKA

\usepackage{xcolor}

%New colors defined below
\definecolor{codegreen}{rgb}{0,0.6,0}
\definecolor{codegray}{rgb}{0.5,0.5,0.5}
\definecolor{codepurple}{rgb}{0.58,0,0.82}
\definecolor{backcolour}{rgb}{0.95,0.95,0.92}

\definecolor{gavron_purple}{HTML}{ac526c}
\definecolor{gavron_blue}{HTML}{bccee6}

%Code listing style named "mystyle"
\lstdefinestyle{mystyle}{
  commentstyle=\color{codegray},
  keywordstyle=\color{magenta},
  numberstyle=\tiny\color{codegray},
  stringstyle=\color{codepurple},
  basicstyle=\ttfamily\footnotesize,
  breaklines=true,                 
  captionpos=t,                    
  keepspaces=true,                 
  numbers=left,                    
  numbersep=5pt,                  
  showtabs=false,                  
  tabsize=2,
}

%"mystyle" code listing set
\lstset{style=mystyle}

\lstdefinelanguage{protobuf}{
keywords={ message, package, int32, float, syntax },
comment=[s]{/*}{*/}, %for multiline comments
}

\lstdefinelanguage{yml}{
keywords={ variables, stage, before_script, script, services, only, image },
comment=[l]{\#}, %use comment-line-style!
string=[s]{"}{"}, %for multiline comments
% morestring=[s]{\$}{:}, %for multiline comments
}

% END OF KOD DODANY PRZEZ BĄCZKA

\newcommand{\listingcaption}[1]% dodane, by można było robić podpis nad dwukolumnowym listingiem
{%
\vspace*{\abovecaptionskip}\small 
\refstepcounter{lstlisting}\hfill%
Listing \thelstlisting: #1\hfill%\hfill%
\addcontentsline{lol}{lstlisting}{\protect\numberline{\thelstlisting}#1}
}%

% Styl zapewniający numerowanie linii
\lstset{
  %%basicstyle=\footnotesize\ttfamily,
  %%columns=fullflexible,
	%%showstringspaces=false,
	%%showspaces=false,
  breaklines=true,
  postbreak=\mbox{\textcolor{red}{$\hookrightarrow$}\space},
  %%numbers=left,
  %%firstnumber=1,
  %%numberfirstline=true,
	%%xleftmargin=17pt,
  %%framexleftmargin=17pt,
  %%framexrightmargin=5pt,
  %%framexbottommargin=4pt,
	belowskip=.5\baselineskip
}

% Styl bez numerownia linii
%%\lstset{
  %%basicstyle=\footnotesize\ttfamily,
  %%columns=fullflexible,
	%%showstringspaces=false,
	%%showspaces=false,
  %%breaklines=true,
  %%postbreak=\mbox{\textcolor{red}{$\hookrightarrow$}\space},
%%}

%% Poniżej sposób ostylowania sposobu podświetlania składni wybranych języków
%%\lstloadlanguages{% Check Dokumentation for further languages ...
%%C,
%%C++,
%%csh,
%%Java
%%}
%%
%%\definecolor{red}{rgb}{0.6,0,0} % for strings
%%\definecolor{blue}{rgb}{0,0,0.6}
%%\definecolor{green}{rgb}{0,0.8,0}
%%\definecolor{cyan}{rgb}{0.0,0.6,0.6}
%%
%%\lstdefinestyle{sqlstyle}{
%%language=SQL,
%%basicstyle=\footnotesize\ttfamily, 
%%numbers=left, 
%%numberstyle=\tiny, 
%%numbersep=5pt, 
%%tabsize=2, 
%%extendedchars=true, 
%%breaklines=true, 
%%showspaces=false, 
%%showtabs=true, 
%%xleftmargin=17pt,
%%framexleftmargin=17pt,
%%framexrightmargin=5pt,
%%framexbottommargin=4pt,
%%keywordstyle=\color{blue}, 
%%commentstyle=\color{green}, 
%%stringstyle=\color{red}, 
%%}
%%
%%\lstdefinestyle{sharpcstyle}{
%%language=[Sharp]C,
%%basicstyle=\footnotesize\ttfamily, 
%%numbers=left, 
%%numberstyle=\tiny, 
%%numbersep=5pt, 
%%tabsize=2, 
%%extendedchars=true, 
%%breaklines=true, 
%%showspaces=false, 
%%showtabs=true, 
%%xleftmargin=17pt,
%%framexleftmargin=17pt,
%%framexrightmargin=5pt,
%%framexbottommargin=4pt,
%%morecomment=[l]{//}, %use comment-line-style!
%%morecomment=[s]{/*}{*/}, %for multiline comments
%%showstringspaces=false, 
%%morekeywords={  abstract, event, new, struct,
                %%as, explicit, null, switch,
                %%base, extern, object, this,
                %%bool, false, operator, throw,
                %%break, finally, out, true,
                %%byte, fixed, override, try,
                %%case, float, params, typeof,
                %%catch, for, private, uint,
                %%char, foreach, protected, ulong,
                %%checked, goto, public, unchecked,
                %%class, if, readonly, unsafe,
                %%const, implicit, ref, ushort,
                %%continue, in, return, using,
                %%decimal, int, sbyte, virtual,
                %%default, interface, sealed, volatile,
                %%delegate, internal, short, void,
                %%do, is, sizeof, while,
                %%double, lock, stackalloc,
                %%else, long, static,
                %%enum, namespace, string},
%%keywordstyle=\color{cyan},
%%identifierstyle=\color{red},
%%stringstyle=\color{blue}, 
%%commentstyle=\color{green},
%%}

\renewcommand\lstlistlistingname{Spis listingów}
\makeatletter
%\renewcommand*{\l@lstlisting}[2]{\@dottedtocline{1}{0em}{2.3em}{#1}{#2}}
\g@addto@macro\insertchapterspace{\addtocontents{lol}{\protect\addvspace{10pt}}}
\renewcommand*{\l@lstlisting}{\@dottedtocline{1}{0em}{2.3em}}
\makeatother

\renewcommand*{\lstlistlistingname}{Spis listingów} \newlistof{lstlistoflistings}{lol}{\lstlistlistingname}



% Choć możliwe jest zastosowanie różnych pakietów formatujących tabele, zaleca się tego nie robić.
%\usepackage{longtable}
%\usepackage{ltxtable}
%\usepackage{tabulary}

%%%%%%%%%%%%%%%%%%%%%%%%%%%%%%%%%%%%%%%%%%%%%%%%%%%
%% Ustawienia odpowiedzialne za sposób łamania dokumentu
%% i ułożenie elementów pływających
%%%%%%%%%%%%%%%%%%%%%%%%%%%%%%%%%%%%%%%%%%%%%%%%%%%
%\hyphenpenalty=10000		% nie dziel wyrazów zbyt często
\clubpenalty=10000      %kara za sierotki
\widowpenalty=10000  % nie pozostawiaj wdów
%\brokenpenalty=10000		% nie dziel wyrazów między stronami - trzeba było wyłączyć, bo nie łamały się linie w lstlisting
%\exhyphenpenalty=999999		% nie dziel słów z myślnikiem - trzeba było wyłączyć, bo nie łamały się linie w lstlisting
\righthyphenmin=3			% dziel minimum 3 litery

%\tolerance=4500
%\pretolerance=250
%\hfuzz=1.5pt
%\hbadness=1450

\renewcommand{\topfraction}{0.95}
\renewcommand{\bottomfraction}{0.95}
\renewcommand{\textfraction}{0.05}
\renewcommand{\floatpagefraction}{0.35}

%%%%%%%%%%%%%%%%%%%%%%%%%%%%%%%%%%%%%%%%%%%%%%%%%%%
%%  Ustawienia rozmiarów: tekstu, nagłówka i stopki, marginesów
%%  dla dokumentów klasy memoir 
%%%%%%%%%%%%%%%%%%%%%%%%%%%%%%%%%%%%%%%%%%%%%%%%%%%
\setlength{\headsep}{10pt} 
\setlength{\headheight}{13.6pt} % wartość baselineskip dla czcionki 11pt tj. \small wynosi 13.6pt
\setlength{\footskip}{\headsep+\headheight}
\setlength{\uppermargin}{\headheight+\headsep+1cm}
\setlength{\textheight}{\paperheight-\uppermargin-\footskip-1.5cm}
\setlength{\textwidth}{\paperwidth-5cm}
\setlength{\spinemargin}{2.5cm}
\setlength{\foremargin}{2.5cm}
\setlength{\marginparsep}{2mm}
\setlength{\marginparwidth}{2.3mm}
%\settrimmedsize{297mm}{210mm}{*}
%\settrims{0mm}{0mm}	
\checkandfixthelayout[fixed] % konieczne, aby się dobrze wszystko poustawiało
%%%%%%%%%%%%%%%%%%%%%%%%%%%%%%%%%%%%%%%%%%%%%%%%
%%  Ustawienia odległości linii, wcięć, odstępów
%%%%%%%%%%%%%%%%%%%%%%%%%%%%%%%%%%%%%%%%%%%%%%%%
\linespread{1}
%\linespread{1.241}
\setlength{\parindent}{14.5pt}
%\setlength{\cftbeforechapterskip}{0.3em} % odstępy w spisie treści
%\setbeforesecskip{10pt plus 0.5ex}%{-3.5ex \@plus -1ex \@minus -.2ex}
%\setaftersecskip{10pt plus 0.5ex}%\onelineskip}
%\setbeforesubsecskip{8pt plus 0.5ex}%{-3.5ex \@plus -1ex \@minus -.2ex}
%\setaftersubsecskip{8pt plus 0.5ex}%\onelineskip}
%\setlength\floatsep{6pt plus 2pt minus 2pt} 
%\setlength\intextsep{12pt plus 2pt minus 2pt} 
%\setlength\textfloatsep{12pt plus 2pt minus 2pt} 

%%%%%%%%%%%%%%%%%%%%%%%%%%%%%%%%%%%%%%%%%%%%%%%%%%%
%%  Pakiety i komendy zastosowane tylko do zamieszczenia informacji o użytych komendach i fontach
%%  Normalnie nie są potrzebne, można je zamarkować podczas redakcji pracy
%%%%%%%%%%%%%%%%%%%%%%%%%%%%%%%%%%%%%%%%%%%%%%%%%%%
\usepackage{memlays}     % extra layout diagrams, zastosowane w szblonie do 'debuggowania', używa pakietu layouts
%\usepackage{layouts}
\usepackage{printlen} % pakiet do wyświetlania wartości zdefiniowanych długości, stosowany do 'debuggowania'
\uselengthunit{pt}
\makeatletter
\newcommand{\showFontSize}{\f@size pt} % makro wypisujące wielkość bieżącej czcionki
\makeatother
% do pokazania ramek można byłoby użyć:
%\usepackage{showframe} 


%%%%%%%%%%%%%%%%%%%%%%%%%%%%%%%%%%%%%%%%%%%%%%%%%%%
%%  Formatowanie list wyliczeniowych, wypunktowań i własnych otoczeń
%%%%%%%%%%%%%%%%%%%%%%%%%%%%%%%%%%%%%%%%%%%%%%%%%%%

% Domyślnie wypunktowania mają zadeklatorowane znaki, które nie występują w tgtermes
% Aby latex nie podstawiał w ich miejsca znaków z czcionki standardowej można zrobić podstawienie:
%    \DeclareTextCommandDefault{\textbullet}{\ensuremath{\bullet}}
%    \DeclareTextCommandDefault{\textasteriskcentered}{\ensuremath{\ast}}
%    \DeclareTextCommandDefault{\textperiodcentered}{\ensuremath{\cdot}}
% Jednak jeszcze lepszym pomysłem jest zdefiniowanie otoczeń z wykorzystaniem enumitem
\usepackage{enumitem} % pakiet pozwalający zarządzać formatowaniem list wyliczeniowych
\setlist{noitemsep,topsep=4pt,parsep=0pt,partopsep=4pt,leftmargin=*} % zadeklarowane parametry pozwalają uzyskać 'zwartą' postać wypunktowania bądź wyliczenia
\setenumerate{labelindent=0pt,itemindent=0pt,leftmargin=!,label=\arabic*.} % można zmienić \arabic na \alph, jeśli wyliczenia mają być z literkami
\setlistdepth{4} % definiujemy głębokość zagnieżdżenia list wyliczeniowych do 4 poziomów
\setlist[itemize,1]{label=$\bullet$}  % definiujemy, jaki symbol ma być użyty w wyliczeniu na danym poziomie
\setlist[itemize,2]{label=\normalfont\bfseries\textendash}
\setlist[itemize,3]{label=$\ast$}
\setlist[itemize,4]{label=$\cdot$}
\renewlist{itemize}{itemize}{4}

%%%http://tex.stackexchange.com/questions/29322/how-to-make-enumerate-items-align-at-left-margin
%\renewenvironment{enumerate}
%{
%\begin{list}{\arabic{enumi}.}
%{
%\usecounter{enumi}
%%\setlength{\itemindent}{0pt}
%%\setlength{\leftmargin}{1.8em}%{2zw} % 
%%\setlength{\rightmargin}{0zw} %
%%\setlength{\labelsep}{1zw} %
%%\setlength{\labelwidth}{3zw} % 
%\setlength{\topsep}{6pt}%
%\setlength{\partopsep}{0pt}%
%\setlength{\parskip}{0pt}%
%\setlength{\parsep}{0em} % 
%\setlength{\itemsep}{0em} % 
%%\setlength{\listparindent}{1zw} % 
%}
%}{
%\end{list}
%}

\makeatletter
\renewenvironment{quote}{
	\begin{list}{}
	{
	\setlength{\leftmargin}{1em}
	\setlength{\topsep}{0pt}%
	\setlength{\partopsep}{0pt}%
	\setlength{\parskip}{0pt}%
	\setlength{\parsep}{0pt}%
	\setlength{\itemsep}{0pt}
	}
	}{
	\end{list}}
\makeatother

%%%%%%%%%%%%%%%%%%%%%%%%%%%%%%%%%%%%%%%%%
%%  Pakiet do generowania indeksu (ważne, aby wstawić przed hyperref)
%%%%%%%%%%%%%%%%%%%%%%%%%%%%%%%%%%%%%%%%%
\DisemulatePackage{imakeidx}
\usepackage[makeindex,noautomatic]{imakeidx} % tutaj mówimy, żeby indeks nie generował się automatycznie, 

%\usepackage[noautomatic]{imakeidx} 
\makeindex

\makeatletter
%%%\renewenvironment{theindex}
							 %%%{\vskip 10pt\@makeschapterhead{\indexname}\vskip -3pt%
								%%%\@mkboth{\MakeUppercase\indexname}%
												%%%{\MakeUppercase\indexname}%
								%%%\vspace{-3.2mm}\parindent\z@%
								%%%\renewcommand\subitem{\par\hangindent 16\p@ \hspace*{0\p@}}%%
								%%%\phantomsection%
								%%%\begin{multicols}{2}
								%%%%\thispagestyle{plain}
								%%%\parindent\z@                
								%%%%\parskip\z@ \@plus .3\p@\relax
								%%%\let\item\@idxitem}
							 %%%{\end{multicols}\clearpage}
%%%
\makeatother


\usepackage{ifpdf}
%\newif\ifpdf \ifx\pdfoutput\undefined
%\pdffalse % we are not running PDFLaTeX
%\else
%\pdfoutput=1 % we are running PDFLaTeX
%\pdftrue \fi
\ifpdf
 \usepackage[pdftex,bookmarks,breaklinks,unicode]{hyperref}
%  WARNING - usunąłem to bo był jakiś "Option clash for package graphicx"   
%  \usepackage[pdftex]{graphicx}
 \DeclareGraphicsExtensions{.pdf,.jpg,.mps,.png}
\pdfcompresslevel=9
\pdfoutput=1
\makeatletter
\AtBeginDocument{  % Poniżej zdefiniowano metadane, jakie zapisane zostaną w dokumencie pdf - należy je właściwie uzupełnić
  \hypersetup{
	pdfinfo={
    Title = {\@title},
    Author = {\@author},
    Subject={},
    Keywords={słowa kluczowe},  
		Producer={producer},
		Creator={pdftex}
	}}
}
\pdftrailerid{} %Remove ID
\pdfsuppressptexinfo15 %Suppress PTEX.Fullbanner and info of imported PDFs

\makeatother
\else
\usepackage{graphicx}
\DeclareGraphicsExtensions{.eps,.ps,.jpg,.mps,.png}
\fi
\sloppy

%\graphicspath{{rys01/}{rys02/}}


%%%%%%%%%%%%%%%%%%%%%%%%%%%%%%%%%%%%%%%%%
% Metadane dla pdfa


%\ifpdf
%\pdfinfo{
   %/Author (Nicola Talbot)
   %/Title  (Creating a PDF document using PDFLaTeX)
   %/CreationDate (D:20040502195600)
   %/ModDate (D:\pdfdate)
   %/Subject (PDFLaTeX)
   %/Keywords (PDF;LaTeX)
%}
%\fi

% Deklaracja głębokościu numeracji
\setcounter{secnumdepth}{2}
\setcounter{tocdepth}{2}
\setsecnumdepth{subsection} % activating subsubsec numbering in doc


% Kropki po numerach sekcji
\makeatletter
\def\@seccntformat#1{\csname the#1\endcsname.\quad}
\def\numberline#1{\hb@xt@\@tempdima{#1\if&#1&\else.\fi\hfil}}
\makeatother

\renewcommand{\chapternumberline}[1]{#1.\quad}
\renewcommand{\cftchapterdotsep}{\cftdotsep}

%\definecolor{niceblue}{rgb}{.168,.234,.671}

% Czcionka do podpisów tabel i rysunków
\captionnamefont{\small}
\captiontitlefont{\small}
% makro pozwalające zmienić sposób wypisywania rozdziału
%\def\printchaptertitle##1{\fonttitle \space \thechapter.\space ##1} 

%\usepackage{ltcaption}
% The ltcaption package supports \CaptionLabelFont & \CaptionTextFont introduced by the NTG document classes
%\renewcommand\CaptionLabelFont{\small}
%\renewcommand\CaptionTextFont{\small}

% Przedefiniowanie etykiet w podpisach tabel i rysunków
%\AtBeginDocument{% 
        \addto\captionspolish{% 
        \renewcommand{\tablename}{Tab.}% 
}%} 

%\AtBeginDocument{% 
%        \addto\captionspolish{% 
%        \renewcommand{\chaptername}{Rozdział}% 
%}} 

%\AtBeginDocument{% 
        \addto\captionspolish{% 
        \renewcommand{\figurename}{Rys.}% 
}%}


%\AtBeginDocument{% 
        \addto\captionspolish{% 
        \renewcommand{\bibname}{Literatura}% 
}%}

%\AtBeginDocument{% 
        \addto\captionspolish{% 
        \renewcommand{\listfigurename}{Spis rysunków}% 
}%}

%\AtBeginDocument{% 
        \addto\captionspolish{% 
        \renewcommand{\listtablename}{Spis tabel}% 
}%}

%\AtBeginDocument{% 
        \addto\captionspolish

%%%%%%%%%%%%%%%%%%%%%%%%%%%%%%%%%%%%%%%%%%%%%%%%%%%%%%%%%%%%%%%%%%                  
%% Definicje stopek i nagłówków
%%%%%%%%%%%%%%%%%%%%%%%%%%%%%%%%%%%%%%%%%%%%%%%%%%%%%%%%%%%%%%%%%%                  
\addtopsmarks{headings}{%
\nouppercaseheads % added at the beginning
}{%
\createmark{chapter}{both}{shownumber}{}{. \space}
%\createmark{chapter}{left}{shownumber}{}{. \space}
\createmark{section}{right}{shownumber}{}{. \space}
}%use the new settings

\makeatletter
\copypagestyle{outer}{headings}
\makeoddhead{outer}{}{}{\small\itshape\rightmark}
\makeevenhead{outer}{\small\itshape\leftmark}{}{}
\makeoddfoot{outer}{\small\@author:~\@titleShort}{}{\small\thepage}
\makeevenfoot{outer}{\small\thepage}{}{\small\@author:~\@title}
\makeheadrule{outer}{\linewidth}{\normalrulethickness}
\makefootrule{outer}{\linewidth}{\normalrulethickness}{2pt}
\makeatother

% fix plain
\copypagestyle{plain}{headings} % overwrite plain with outer
\makeoddhead{plain}{}{}{} % remove right header
\makeevenhead{plain}{}{}{} % remove left header
\makeevenfoot{plain}{}{}{}
\makeoddfoot{plain}{}{}{}

\copypagestyle{empty}{headings} % overwrite plain with outer
\makeoddhead{empty}{}{}{} % remove right header
\makeevenhead{empty}{}{}{} % remove left header
\makeevenfoot{empty}{}{}{}
\makeoddfoot{empty}{}{}{}


%%%%%%%%%%%%%%%%%%%%%%%%%%%%%%%%%%%%%%%
%% Definicja strony tytułowej 
%%%%%%%%%%%%%%%%%%%%%%%%%%%%%%%%%%%%%%%
\makeatletter
%Uczelnia
\newcommand\uczelnia[1]{\renewcommand\@uczelnia{#1}}
\newcommand\@uczelnia{}
%Wydział
\newcommand\wydzial[1]{\renewcommand\@wydzial{#1}}
\newcommand\@wydzial{}
%Kierunek
\newcommand\kierunek[1]{\renewcommand\@kierunek{#1}}
\newcommand\@kierunek{}
%Specjalność
\newcommand\specjalnosc[1]{\renewcommand\@specjalnosc{#1}}
\newcommand\@specjalnosc{}
%Tytuł po angielsku
\newcommand\titleEN[1]{\renewcommand\@titleEN{#1}}
\newcommand\@titleEN{}
%Tytuł krótki
\newcommand\titleShort[1]{\renewcommand\@titleShort{#1}}
\newcommand\@titleShort{}
%Promotor
\newcommand\promotor[1]{\renewcommand\@promotor{#1}}
\newcommand\@promotor{}

%\usepackage[absolute]{textpos} % zamarkowano, bo ostatecznie wykorzystano otoczenie picture

\def\maketitle{%
  \pagestyle{empty}%
%%\garamond 
	\fontfamily{\ebgaramond@family}\selectfont % na stronie tytułowej czcionka garamond
%%%%%%%%%%%%%%%%%%%%%%%%%%%%%%%%%%%%%	
%% Poniżej, w otoczniu picture, wstawiono tytuł i autora. 
%% Tytuł (z autorem) musi znaleźć się w obszarze 
%% odpowiadającym okienku 110mmx75mm, którego lewy górny róg 
%% jest w położeniu 77mm od lewej i 111mm od górnej  krawędzi strony 
%% (tak wynika z wycięcia na okładce). 
%% Poniższy kod musi być użyty dokładnie w miejscu gdzie jest.
%% Jeśli tytuł nie mieści się w okienku, to należy tak pozmieniać 
%% parametry użytych komend, aby ten przydługi tytuł jednak 
%% upakować go do okienka.
%%
%% Sama okładka (kolorowa strona z wycięciem, do pobrania z dydaktyki) 
%% powinna być przycięta o 3mm od każdej z krawędzi.
%% Te 3mm pewnie zostawiono na ewentualne spady czy też specjalną oprawę.
%%%%%%%%%%%%%%%%%%%%%%%%%%%%%%%%%%%%%	
\newlength{\tmpfboxrule}
\setlength{\tmpfboxrule}{\fboxrule}
\setlength{\fboxsep}{2mm}
\setlength{\fboxrule}{0mm} 
%\setlength{\fboxrule}{0.1mm} %% jeśli chcemy zobaczyć ramkę
\setlength{\unitlength}{1mm}
\begin{picture}(0,0)
\put(26,-124){\fbox{
\parbox[c][71mm][c]{104mm}{\centering%\lineskip=34pt 
\fontsize{16pt}{18pt}\selectfont \@title\\[5mm]
\fontsize{16pt}{18pt}\selectfont \@titleEN\\[20mm]
\fontsize{16pt}{18pt}\selectfont AUTOR:\\[2mm]
\fontsize{14pt}{16pt}\selectfont \@author}
}
}
\end{picture}
\setlength{\fboxrule}{\tmpfboxrule} 
%%%%%%%%%%%%%%%%%%%%%%%%%%%%%%%%%%%%%
%% Reszta strony z nazwą uczelni, wydziału, kierunkiem, specjalnością
%% promotorem, oceną pracy, miastem i rokiem
	{\centering%\vspace{-1cm}
		{\fontsize{22pt}{24pt}\selectfont \@uczelnia}\\[0.4cm]
		{\fontsize{22pt}{24pt}\selectfont \@wydzial}\\[0.5cm]
		  \hrule %\vspace*{0.7cm}
	}
{\flushleft\fontsize{14pt}{16pt}\selectfont%
\begin{tabular}{ll}
KIERUNEK: & \@kierunek\\
SPECJALNOŚĆ: & \@specjalnosc\\
\end{tabular}\\[1.3cm]
}
{\centering
{\fontsize{32pt}{36pt}\selectfont PRACA DYPLOMOWA}\\[0.5cm]
{\fontsize{32pt}{36pt}\selectfont INŻYNIERSKA}\\[2.5cm]
}
\vfill
\begin{tabularx}{\linewidth}{p{6cm}X}
		&{\fontsize{16pt}{18pt}\selectfont PROWADZĄCY PRACĘ:}\\[2mm] %UWAGA: tutaj jest miejsce na nazwisko promotora pracy
		&{\fontsize{14pt}{16pt}\selectfont \@promotor}\\[10mm]
		&{\fontsize{16pt}{18pt}\selectfont OCENA PRACY:}\\[20mm]
	\end{tabularx}
\vspace{2cm}
\hrule\vspace*{0.3cm}
{\centering
{\fontsize{16pt}{18pt}\selectfont \@date}\\[0cm]
}
%\ungaramond
\normalfont
 \cleardoublepage
}
\makeatother
%%%%%%%%%%%%%%%%%%%%%%%%%%%%%%%%%%%%%%%%%

%\AtBeginDocument{\addtocontents{toc}{\protect\thispagestyle{empty}}}




%%%%%%%%%%%%%%%%%%%%%%%%%%%%%%%%%%%%%%%%%
%%  Metadane dokumentu 
%%%%%%%%%%%%%%%%%%%%%%%%%%%%%%%%%%%%%%%%%
\title{System inspekcji obszarów z wykorzystaniem autonomicznych dronów}
\titleShort{System inspekcji obszarów ...}
\titleEN{Autonomous drone-based scouting system}
\author{Mateusz Bączek}
\uczelnia{POLITECHNIKA WROCŁAWSKA}
\wydzial{WYDZIAŁ ELEKTRONIKI}
\kierunek{INFORMATYKA}
\specjalnosc{SYSTEMY INFORMATYKI W MEDYCYNIE}
\promotor{Dr inż. Michał Kucharzak, Katedra Systemów i Sieci Komputerowych}
\date{WROCŁAW, 2020}

% Ustawienie odstępu od góry w nienumerowanych rozdziałach oraz wykazach:
% Spis treści, Spis tabel, Spis rysunków, Indeks rzeczowy

%\newlength{\linespace}
%\setlength{\linespace}{-\beforechapskip-\topskip+\headheight+\topsep}
%\makechapterstyle{noNumbered}{%
%\renewcommand\chapterheadstart{\vspace*{\linespace}}
%}

%% powyższa komenda załatwia to, co robią komendy poniższe dla spisów
%\renewcommand*{\tocheadstart}{\vspace*{\linespace}}
%\renewcommand*{\lotheadstart}{\vspace*{\linespace}}
%\renewcommand*{\lofheadstart}{\vspace*{\linespace}}

%%%%%%%%%%%%%%%%%%%%%%%%%%%%%%%%%%%%%%%%%
%                  Początek dokumentu 
%%%%%%%%%%%%%%%%%%%%%%%%%%%%%%%%%%%%%%%%%
%\includeonly{skroty,rozdzial01} % jeśli chcemy kompilować tylko fragmenty, to można tu je wpisać

\begin{document}
% Tutaj można przełączyć odstęp między liniami
% \SingleSpacing
\OnehalfSpacing
% \DoubleSpacing
%\settypeoutlayoutunit{cm} % do debugowania
%\typeoutstandardlayout    % wypisuje na stdout informacje o ustawieniach
\pdfbookmark[0]{Tytuł}{Tytul.1}
\maketitle

% \newpage
% \thispagestyle{empty}
% \mbox{}\vfill
% \noindent\begin{tabular}{@{}ll} Opracował: & Tomasz Kubik \texttt{<tomasz.kubik@pwr.edu.pl>}\\
%  Data: & marzec 2020 
%  \end{tabular}\\[15mm]
% \noindent\includegraphics[width=3cm]{by-nc-sa}\newline
% {\normalfont 
% Tekst zawarty w niniejszym szablonie jest udostępniany na licencji Creative Commons: \emph{Uznanie autorstwa -- Użycie niekomercyjne -- Na tych samych warunkach, 3.0 Polska}, Wrocław 2020. \\[2pt]
% Oznacza to, że wszystkie przekazane treści można kopiować i  wykorzystywać do celów niekomercyjnych, a także tworzyć na ich podstawie utwory zależne pod warunkiem podania autora i~nazwy licencjodawcy oraz udzielania na utwory zależne takiej samej licencji. Tekst licencji jest dostępny pod adresem: \url{http://creativecommons.org/licenses/by-nc-sa/3.0/pl/}.\\[2pt]
% Podczas redakcji pracy dyplomowej poniższą stronę można usunąć (CC dotyczy tekstu, a nie samego latexowego szablonu)
% }
\newpage


\chapterstyle{noNumbered}
\pagestyle{outer}
\mbox{}\pdfbookmark[0]{Spis treści}{spisTresci.1}

\setcounter{tocdepth}{1}
\tableofcontents* 


% TODO: UNCOMMENT

\newpage
\mbox{}\pdfbookmark[0]{Spis rysunków}{spisRysunkow.1}
\addcontentsline{toc}{chapter}{Spis rysunków}
\listoffigures*

\newpage
\mbox{}\pdfbookmark[0]{Spis listingów}{spisListingow.1}
\addcontentsline{toc}{chapter}{Spis listingów}
\lstlistoflistings*

% END OF TODO: UNCOMMENT

% nie wiem do czego to jest, na razie zakomentowane
% {%
% \let\oldnumberline\numberline%
% \renewcommand{\numberline}{\figurename~\oldnumberline}%
% \listoffigures%
% }


% TODO: UNCOMMENT

\newpage
\mbox{}\pdfbookmark[0]{Spis tabel}{spisTabel.1}
\addcontentsline{toc}{chapter}{Spis tabel}
\listoftables*
\chapter*{Skróty}\mbox{}\pdfbookmark[0]{Skróty}{skroty.1}
\label{sec:skroty}
\noindent
\begin{description}[labelwidth=*]
  \item [GCS] (ang.\ \emph{Ground control station})
  \item [JSON] (ang.\ \emph{JavaScript Object Notation})
  \item [Protobuf] (ang.\ \emph{Protocol Buffers})
\end{description}
 %skróty można sobie pominąć
% END OF TODO: UNCOMMENT

\chapterstyle{default}

\chapter{Wstęp} \label{chapter_intro}


\section{Geneza pracy} \label{intro_genesis}
% kilka akapitów wprowadzających,
%~z czego powinno wynikać, że zrobienie
% takiego systemu jest potrzebne ~i$3ma sens.

Sektor gospodarki związany~z inżynierią bezzałogowych maszyn lotniczych znajduje się~w
stanie dynamicznego rozwoju~i generuje coraz większe przychody.
Tym samym przyciąga inwestorów skłonnych zainwestować~w innowacyjne pomysły zarówno 
dużych przedsiębiorstw, jak~i młodych konstruktorów. Międzynarodowe zawody sponsorowane
przez podmioty prywatne~i organizacje rządowe umożliwiają pasjonatom wdrożenie~w życie
ich wizji. Przykład stanowi \textit{UAV Challenge} -- zawody skoncentrowane na
autonomicznych systemach wsparcia służb medycznych, współorganizowane przez rząd
australijskiego stanu Queensland, które przyciągają corocznie najlepsze zespoły~z całego
świata\cite{uav_sponsors}. Potencjał inwestycyjny~i naukowy, którym dysponuje ta branża,
sprzyja powstawaniu różnorodnych rozwiązań.~W konsekwencji bezpośrednio niezwiązane~z lotnictwem sektory korzystają~z dorobku autonomicznej awiacji.
Tytułem przykładu: generowanie trójwymiarowych map terenu, przy użyciu zdjęć wykonanych~z dronów znajduje
zastostowanie~w dziedzinach takich jak górnictwo, geodezja czy militaria
\cite{uav_photogrametry}, natomiast dostarczanie towarów indywidualnym klientom~z pomocą
dronów jest~w stanie zrewolucjonizować przemysł transportowy \cite{prime_air}.

System realizowany~w ramach pracy mógłby potencjalnie służyć jako wsparcie dla służb
medycznych, wspomagając poszukiwanie osób zaginionych. Jest jednak to jedynie pojedynczy 
przykład z szerokiego spektrum potencjalnych zastosowań. Algorytmy sztucznej inteligencji,
zastosowane do analizy zdjęć wykonanych~w czasie lotów, mogą zostać wytrenowane
na nowych danych, aby rozpoznawać dowolne obiekty (samochody, budynki, drzewa). 
Dzięki temu, system może zostać szybko przystosowany do nowych celów.

Budowa systemu wykorzystującego fizyczne komponenty, takie jak autonomiczne drony,
wymaga wprowadzenia dodatkowych zabezpieczeń, pozwalających na przetestowanie stabilności
systemu, zanim zostanie uruchomiony w terenie. Wymaganie podyktowane jest wysokim kosztem
dronów oraz względami bezpieczeństwa -- utrata kontroli nad maszyną latającą, gdy wokół znajdują
się ludzie, stanowi duże zagrożenie. Nowoczesne praktyki inżynierii oprogramowania, takie jak 
automatyczne testy w potoku \textit{CI/CD} oraz zastosowanie symulatorów i atrap
w miejsce rzeczywistych, fizycznych obiektów pozwalają na budowę takich zabezpieczeń.

Przedstawiony~w pracy prototyp systemu inspekcji obszarów powstał~w ramach działalności
Akademickiego Klubu Lotniczego -- koła naukowego Politechniki Wrocławskiej, zajmującego
się rozwijaniem technologii związanych~z autonomicznymi dronami~i samolotami
\cite{akl_home_page}. Projekt został sfinansowany przez Wrocławski Fundusz
Aktywności Studenckiej \cite{fast_webpage}.

% Szczególnie interesujące są zagadnienia integracji komponentów systemu,
% oraz testowanie - które~w przypadku systemu angażującego rzeczywiste
% maszyny nie może ograniczyć się jedynie do standardowych testów jednostkowych.

% \newpage
\section{Cel pracy} \label{intro_objective}
% koniecznie sformułowania:
% - Celem ogólnym pracy jest…?,
% - Celami szczegółowymi pracy są…?

Celem pracy jest zaprojektowanie architektury i budowa prototypu systemu inspekcji obszarów,
wykorzystującego autonomiczne drony. %realizowanego w ramach działalności koła naukowego Akademicki Klub Lotniczy.
% TODO: FIX SYSTEM MA WYKORZYSTYWAĆ
System ma wykorzystywać napisaną~w ramach pracy
infrastrukturę informatyczną, pozwalającą na planowanie tras lotów, obsługę telemetrii~i
rozpoznawanie obiektów na zdjęciach wykonanych~w czasie lotu, za pomocą algorytmów
sztucznej inteligencji. 

Architektura systemu (wykorzystane technologie~i struktura podprojektów składających
się na system) musi pozwalać na zautomatyzowanie procesu wdrażania
systemu, oraz zautomatyzowanie wdrażania nowych funkcjonalności -- każde~z wdrożeń musi
być poprzedzone testami integracyjnymi na poziomie całego systemu.
Aby przeprowadzić pełne testy integracyjne, prototyp systemu będzie uruchamiany~w
środowisku testowym, zawierającym zintegrowany symulator drona/samolotu.

Finalnie, system ma pozwolić użytkownikowi na zaplanowanie trasy przelotu drona
oraz wyznaczenie harmonogramu, według którego mają odbywać się loty.~W trakcie 
lotu, system ma~w czasie rzeczywistym dostarczać użytkownikowi dane telemetryczne~o
obecnej pozycji maszyny~i pozwalać na podgląd zdjęć wykonywanych~w czasie lotu.
Po wylądowaniu, system wygeneruje dla użytkownika podsumowanie lotu, zawierające
trasę pokonaną przez drona, wraz~z wykonanymi zdjęciami, na których oznaczone zostaną
rozpoznane przez algorytmy sztucznej inteligencji obiekty.

\section{Zakres~i struktura pracy} \label{intro_scope}

Zakres pracy obejmuje elementy projektu związane~z inżynierią~i architekturą
oprogramowania -- proces projektowania struktury systemu,
wybór technologii, zaprojektowanie punktów stykowych~w systemie, automatyzacja
procesu wdrażania~i testowania systemu.

W rozdziale \ref{chapter_functional_requirements} opisane są wymagania funkcjonalne
każdego z komponentów systemu. Następnie, rozdział \ref{chapter_architecture} opisuje
proces wyboru technologii i frameworków zastosowanych do budowy systemu. 
Rozwiązania wykorzystane w celu automatyzacji budowania i wdrażania systemu opisane
są w rozdziale \ref{chapter_deployment}. Testy systemu (zarówno integracyjne jak
i w terenie) opisuje rozdział~\ref{chapter_tests}.

\chapter{Wymagania funkcjonalne systemu} \label{chapter_functional_requirements}

\section{Oprogramowanie na dronie}

Wielowirnikowce podłączone do systemu muszą być
zdolne do autonomicznego lotu -- w kontekście pracy
oznacza to zdolność do automatycznego startu,
lądowania, stabilizacji oraz samodzielnego
lotu do koordynatów GPS. W trakcie lotu, maszyna musi
zbierać i wysyłać dwa rodzaje danych:
\begin{itemize}
    \item dane telemetryczne,
    \item zdjęcia wykonane w czasie lotu.
\end{itemize}

Dane telemetryczne muszą zawierać informacje o pozycji
drona oraz identyfikować maszynę za pomocą unikatowego
identyfikatora oraz numeru lotu. Przesyłany jest też
poziom naładowania baterii oraz informacja, czy w obecnym
czasie prowadzone jest nagrywanie.

\section{Protokoły wymiany danych}

W przypadku systemu działającego autonomicznie, wymiana danych jest kluczowym
elementem pozwalającym na sprawdzanie poprawności działania
i diagnozowania błędów w systemie. Podczas lotów testowych
często nie ma możliwości bezpośredniej obserwacji systemu
lub ingerencji w jego sposób działania. Odpowiednia architektura
zbierająca i archiwizująca dane z lotów pozwala znacznie szybciej 
wykryć potencjalne problemy i zapobiec krytycznym błędom. 

\subsection{Dane telemetryczne}
Protokół do wymiany danych telemetrycznych powinien
wysyłać dane w postaci binarnej, gdyż jest to bardziej
efektywne niż kodowanie ich w postaci tekstowej (na
przykład w formacie \texttt{JSON}, typowym dla języków
wysokopoziomowych - wykorzystywanym w technologiach webowych).

Protokół powinien być w łatwy sposób rozszerzalny,
pozwalając w przyszłości zredefiniować część wysyłanych
pakietów lub dodać nowe dane, bez tracenia kompatybilności
wstecznej bądź konieczności przebudowania całego systemu.
Poszczególne komponenty systemu będą pisane w różnych
językach programowania -- biorąc to pod uwagę, pożądaną
cechą protokołu jest też możliwość szybkiego przeportowania
go na inny język programowania. 

\subsection{Zdjęcia wykonane w trakcie lotu}
Efektywny przesył zdjęć oraz filmów to temat zbyt obszerny
i wymagający, żeby poruszać go w treści pracy - system
powinien wykorzystywać dowolny prosty w implementacji
protokół wysyłania zdjęć. Architektura systemu powinna
zapewnić możliwość prostej wymiany tego komponentu, dzięki
czemu w przyszłości będzie możliwe zastąpienie go
przez bardziej zoptymalizowane rozwiązanie.

\section{Oprogramowanie serwerowe}

Serwer webowy jest komponentem, który odbiera, archiwizuje i przekazuje
dane nadchodzące z dronów do aplikacji klienckiej. Jest punktem centralnym systemu,
wykorzystywanym bezpośrednio przez wszystkie pozostałe elementy.

\subsection{Odbiór i multipleksowanie telemetrii}

Aby umożliwić diagnozowanie stanu systemu w czaie rzeczywistym -- szczególnie
stanu wykonujących lot wielowirnikowców, telemetria nadchodząca z maszyn nie może być
jedynie archiwizowana na dysku serwera centralnego. Konieczną funkcjonalnością jest
przesyłanie jej w czasie rzeczywistym do wielu jednocześnie podłączonych klientów.

Umożliwi to podjęcie akcji w przypadku wykrycia krytycznego błędu, który mógłby
zakończyć się uszkodzeniem bądź rozbiciem drona, ułatwi też wykonywanie testów - zarówno
na rzeczywistych maszynach, jak i wykorzystujących symulatory lotu.

\section{Oprogramowanie klienckie}

Aplikacja kliencka skupiona jest wokół trzech funkcjonalności:

\begin{enumerate}
    \item planowanie tras i harmonogramu lotów,
    \item odbiór telemetrii i zdjęć z dronów w czasie rzeczywistym,
    \item przegląd i analiza telemetrii oraz zdjęć zarchiwizowanych z poprzednich lotów.
\end{enumerate}

Kluczowym elementem aplikacji klienckiej jest obsługa mapy -- wszystkie
wymienione funkcjonalności wymagają wizualizacji nadchodzących danych geograficznych,
rozszerzonych o dodatkowe informacje (na przykład godzina przelotu przez dany punkt lub
zarejestrowane w danym miejscu zdjęcia i wykryte na nich obiekty).
\chapter{Redakcja pracy}
\section{Układ pracy}
Standardowo praca powinna być zredagowana w następującym układzie:

\noindent\fbox{\begin{minipage}{\dimexpr\textwidth-2\fboxsep-2\fboxrule\relax}
\begin{quote}
\item Strona tytułowa
\item Strona z dedykacją (opcjonalna)
\item Spis treści  
\item Spis rysunków (opcjonalny)
\item Spis tabel (opcjonalny)
\item Skróty (wykaz opcjonalny)
\item 1. Wstęp 
\begin{quote}
\item 1.1 Cel i zakres pracy 
\item 1.2 Układ pracy 
\end{quote}
\item 2. Kolejny rozdział
\begin{quote}
\item 2.1 Sekcja
\begin{quote}
\item 2.1.1 Podsekcja
\begin{quote}
\item Nienumerowana podpodsekcja
\begin{quote}
\item Paragraf
\end{quote}
\end{quote}
\end{quote}
\end{quote}
\item $\ldots$
\item \#. Podsumownie i wnioski
\item Literatura
\item A. Dodatek
\begin{quote}
\item A.1 Sekcja w dodatku
\end{quote}
\item $\ldots$
\item \$. Zawartość płyty CD/DVD
\item Indeks rzeczowy (opcjonalny)
\end{quote}
\end{minipage}}\\

Spis treści -- powinien być generowany automatycznie, z podaniem tytułów i numerów stron. Typ czcionki oraz wielkość liter spisu treści powinny być takie same jak w niniejszym wzorcu.

Spis rysunków, Spis tabel -- powinny być generowane automatycznie (podobnie jak Spis treści). Elementy te są opcjonalne (robienie osobnego spisu, w którym na przykład są tylko dwie pozycje specjalnie nie ma sensu).

Wstęp -- pierwszy rozdział, w którym powinien znaleźć się opis dziedziny, w jakiej osadzona jest praca, oraz wyjaśnienie motywacji do podjęcia tematu.  
W sekcji ,,Cel i zakres'' powinien znaleźć się opis celu oraz zadań do wykonania, zaś w sekcji ,,Układ pracy'' -- opis zawartości kolejnych rozdziałów.

Podsumowanie -- w rozdziale tym powinny być zamieszczone: podsumowanie uzyskanych efektów oraz wnioski końcowe wynikające z realizacji celu pracy dyplomowej.

Literatura -- wykaz źródeł wykorzystanych w pracy (do każdego źródła musi istnieć odpowiednie cytowanie w tekście). Wykaz ten powinien być generowany automatycznie.

Dodatki -- miejsce na zamieszczanie informacji dodatkowych, jak: Instrukcja wdrożeniowa, Instrukcja uruchomieniowa, Podręcznik użytkownika itp.
Osobny dodatek powinien być przeznaczony na opis zawartości dołączonej płyty CD/DVD. Założono, że będzie to zawsze ostatni dodatek.

Indeks rzeczowy -- miejsce na zamieszczenie kluczowych wyrazów, do których czytelnik będzie chciał sięgnąć. Indeks powinien być generowany automatycznie. Jego załączanie jest opcjonalne.
\section{Styl}
\label{sec:Styl}
Zasady pisania pracy (przy okazji można tu zaobserwować efekt wyrównania wpisów występujących na liście wyliczeniowej uzależnione od długości etykiety):
\begin{enumerate}[labelwidth=\widthof{\ref{last-item}},label=\arabic*.]
\item Praca dyplomowa powinna być napisana w  formie bezosobowej (,,w pracy pokazano ...''). Taki styl przyjęto na uczelniach w naszym kraju, choć w krajach anglosaskich preferuje się redagowanie treści w pierwszej osobie.
\item W tekście pracy można odwołać się do myśli autora, ale nie w pierwszej osobie, tylko poprzez wyrażenia typu: ,,autor wykazał, że ...''. 
\item Odwołując się do rysunków i tabel należy używać zwrotów typu: ,,na rysunku pokazano ...'', ,,w tabeli zamieszczono ...'' (tabela i rysunek to twory nieżywotne, więc ,,rysunek pokazuje'' jest niepoprawnym zwrotem).
\item Praca powinna być napisana językiem formalnym, bez wyrażeń żargonowych (,,sejwowanie'' i ,,downloadowanie''), nieformalnych czy zbyt ozdobnych (,,najznamienitszym przykładem tego niebywałego postępu ...'')
\item Pisząc pracę należy dbać o poprawność stylistyczną wypowiedzi
\begin{itemize}
\item trzeba pamiętać, do czego stosuje się ,,liczba'', a do czego ,,ilość'',
\item nie ,,szereg funkcji'' tylko ,,wiele funkcji'',
\item redagowane zdania nie powinny być zbyt długie (lepiej podzielić zdanie wielokrotnie złożone na pojedyncze zdania),
\item itp.
\end{itemize}
\item Zawartość rozdziałów powinna być dobrze wyważona. Nie wolno więc generować sekcji i podsekcji, które mają zbyt mało tekstu lub znacząco różnią się objętością. Zbyt krótkie podrozdziały można zaobserwować w przykładowym rozdziale~\ref{chap:podsumowanie}.
\item Niedopuszczalne jest pozostawienie w pracy błędów ortograficznych czy tzw.\ literówek -- można je przecież znaleźć i skorygować
automatycznie. \addtocounter{enumi}{9997} 
\item  Niedopuszczalne jest pozostawienie w pracy błędów ortograficznych czy tzw.\ literówek -- można je przecież znaleźć i skorygować
automatycznie. \label{last-item}
\end{enumerate}




\chapter{Wybór technologii i architektura systemu}

\section{Struktura repozytoriów}
\section{Oprogramowanie na dronie}
\section{Protokoły wymiany danych}
\section{Oprogramowanie serwerowe}
\section{Oprogramowanie klienckie}

% \section{Praca z wieloma repozytoriami}
\section{Wspólne punkty stykowe - \texttt{git submodules}}

% \chapter{Metodyka pracy i zarządzania}
% \section{Efektywne wykorzystanie narzędzi dostępnych w popularnych systemach kontroli wersji}

% ================================== % 
\chapter{Wdrażanie systemu}

\section{Konteneryzacja}
\section{Automatyczne budowanie projektów}
\section{Automatyczne aktualizacje kontenerów}
% \section{Automatyczne wdrażanie statycznego kodu}


% ================================== % 
\chapter{Testy systemu}

\section{Testy jednostkowe}
\section{Testy integracyjne}
\subsection{Symulacja i symulatory}
\section{Systemy ciągłej integracji}
\section{Testy w terenie}

% ================================== % 
\chapter{Podsumowanie}

\section{Wyniki testów}
\section{Osiągnięta sprawność}
\section{Pola do poprawy}
\section{Wnioski}

%%%2. środowisko do pisania kodu latexa: 
%%%( )
%%%3. viewer pdf-ów, pozwalający na nawigację zwrotną: Sumatra PDF 3.0
%%%(http://www.sumatrapdfreader.org/download-free-pdf-viewer.html)
%%%
%%%- o konfiguracji texniccenter do współdziałania z sumatra pdf można poczytać sobie na stronie:
%%%http://tex.stackexchange.com/questions/116981/how-to-configure-texniccenter-2-0-with-sumatra-2013-2014-2015-version
%%%(można znaleźć też inne tutoriale)
%%%
%%%4. środowisko do zarządzania bibliografią: JabRef
%%%(http://jabref.sourceforge.net/download.php)
%%%
%%%Polecam też instalację pod windowsami następujących narzędzi:
%%%- Sumatra PDF - przeglądarka pdf umożliwiająca nawigację pomiędzy
%%%edytowanym tekstem a przeglądanym dokumentem (podglądanie tekstu w
%%%TeXnicCenter umieszcza kursor w odpowiednim miejscu w pdfie, podwójne
%%%kliknięcie w pdfie ustawia kursor w edytorze tekstu).
%%%- JabRef - narzędzie do przygotowywania bibliografii.
%%%
%%%
%%%Uwaga: tytuł powinien zmieścić się w okienku kolorowej okładki (którą
%%%powinna dostarczyć uczelniana administracja). Proszę posterować
%%%parametrami, aby "wpasować" w okienko własny tekst.
%%%
%%%Do ASAPa należy wprowadzić pracę dyplomową/projekt inżynierski w pliku o nazwie:
%%%
%%%W04_[nr albumu]_[rok kalendarzowy]_[rodzaj pracy] (szczegółowa instrukcja pod adresem asap.pwr.edu.pl)
%%%
           %%%Przykładowo:
        %%%­W04_123456_2015_praca inżynierska.pdf     - praca dyplomowa inżynierska
        %%%W04_123456_2015_projekt inżynierski.pdf   - projekt inżynierski
        %%%W04_123456_2015_praca magisterska.pdf  - praca dyplomowa magisterska
%%%
              %%%rok kalendarzowy ? rok realizacji kursu „Praca dyplomowa” (nie rok obrony) 

% \chapter{Redakcja pracy}
\section{Układ pracy}
Standardowo praca powinna być zredagowana w następującym układzie:

\noindent\fbox{\begin{minipage}{\dimexpr\textwidth-2\fboxsep-2\fboxrule\relax}
\begin{quote}
\item Strona tytułowa
\item Strona z dedykacją (opcjonalna)
\item Spis treści  
\item Spis rysunków (opcjonalny)
\item Spis tabel (opcjonalny)
\item Skróty (wykaz opcjonalny)
\item 1. Wstęp 
\begin{quote}
\item 1.1 Cel i zakres pracy 
\item 1.2 Układ pracy 
\end{quote}
\item 2. Kolejny rozdział
\begin{quote}
\item 2.1 Sekcja
\begin{quote}
\item 2.1.1 Podsekcja
\begin{quote}
\item Nienumerowana podpodsekcja
\begin{quote}
\item Paragraf
\end{quote}
\end{quote}
\end{quote}
\end{quote}
\item $\ldots$
\item \#. Podsumownie i wnioski
\item Literatura
\item A. Dodatek
\begin{quote}
\item A.1 Sekcja w dodatku
\end{quote}
\item $\ldots$
\item \$. Zawartość płyty CD/DVD
\item Indeks rzeczowy (opcjonalny)
\end{quote}
\end{minipage}}\\

Spis treści -- powinien być generowany automatycznie, z podaniem tytułów i numerów stron. Typ czcionki oraz wielkość liter spisu treści powinny być takie same jak w niniejszym wzorcu.

Spis rysunków, Spis tabel -- powinny być generowane automatycznie (podobnie jak Spis treści). Elementy te są opcjonalne (robienie osobnego spisu, w którym na przykład są tylko dwie pozycje specjalnie nie ma sensu).

Wstęp -- pierwszy rozdział, w którym powinien znaleźć się opis dziedziny, w jakiej osadzona jest praca, oraz wyjaśnienie motywacji do podjęcia tematu.  
W sekcji ,,Cel i zakres'' powinien znaleźć się opis celu oraz zadań do wykonania, zaś w sekcji ,,Układ pracy'' -- opis zawartości kolejnych rozdziałów.

Podsumowanie -- w rozdziale tym powinny być zamieszczone: podsumowanie uzyskanych efektów oraz wnioski końcowe wynikające z realizacji celu pracy dyplomowej.

Literatura -- wykaz źródeł wykorzystanych w pracy (do każdego źródła musi istnieć odpowiednie cytowanie w tekście). Wykaz ten powinien być generowany automatycznie.

Dodatki -- miejsce na zamieszczanie informacji dodatkowych, jak: Instrukcja wdrożeniowa, Instrukcja uruchomieniowa, Podręcznik użytkownika itp.
Osobny dodatek powinien być przeznaczony na opis zawartości dołączonej płyty CD/DVD. Założono, że będzie to zawsze ostatni dodatek.

Indeks rzeczowy -- miejsce na zamieszczenie kluczowych wyrazów, do których czytelnik będzie chciał sięgnąć. Indeks powinien być generowany automatycznie. Jego załączanie jest opcjonalne.
\section{Styl}
\label{sec:Styl}
Zasady pisania pracy (przy okazji można tu zaobserwować efekt wyrównania wpisów występujących na liście wyliczeniowej uzależnione od długości etykiety):
\begin{enumerate}[labelwidth=\widthof{\ref{last-item}},label=\arabic*.]
\item Praca dyplomowa powinna być napisana w  formie bezosobowej (,,w pracy pokazano ...''). Taki styl przyjęto na uczelniach w naszym kraju, choć w krajach anglosaskich preferuje się redagowanie treści w pierwszej osobie.
\item W tekście pracy można odwołać się do myśli autora, ale nie w pierwszej osobie, tylko poprzez wyrażenia typu: ,,autor wykazał, że ...''. 
\item Odwołując się do rysunków i tabel należy używać zwrotów typu: ,,na rysunku pokazano ...'', ,,w tabeli zamieszczono ...'' (tabela i rysunek to twory nieżywotne, więc ,,rysunek pokazuje'' jest niepoprawnym zwrotem).
\item Praca powinna być napisana językiem formalnym, bez wyrażeń żargonowych (,,sejwowanie'' i ,,downloadowanie''), nieformalnych czy zbyt ozdobnych (,,najznamienitszym przykładem tego niebywałego postępu ...'')
\item Pisząc pracę należy dbać o poprawność stylistyczną wypowiedzi
\begin{itemize}
\item trzeba pamiętać, do czego stosuje się ,,liczba'', a do czego ,,ilość'',
\item nie ,,szereg funkcji'' tylko ,,wiele funkcji'',
\item redagowane zdania nie powinny być zbyt długie (lepiej podzielić zdanie wielokrotnie złożone na pojedyncze zdania),
\item itp.
\end{itemize}
\item Zawartość rozdziałów powinna być dobrze wyważona. Nie wolno więc generować sekcji i podsekcji, które mają zbyt mało tekstu lub znacząco różnią się objętością. Zbyt krótkie podrozdziały można zaobserwować w przykładowym rozdziale~\ref{chap:podsumowanie}.
\item Niedopuszczalne jest pozostawienie w pracy błędów ortograficznych czy tzw.\ literówek -- można je przecież znaleźć i skorygować
automatycznie. \addtocounter{enumi}{9997} 
\item  Niedopuszczalne jest pozostawienie w pracy błędów ortograficznych czy tzw.\ literówek -- można je przecież znaleźć i skorygować
automatycznie. \label{last-item}
\end{enumerate}



% \chapter{Testy systemu} \label{chapter_tests}

Jak zaznaczono w celu pracy (\ref{intro_objective}), prototyp systemu
ma być w pełni testowalny. Testy wykonywane są wewnątrz systemu ciągłej integracji 
(\textit{CI/CD}), opisanym w rozdziale \ref{chapter_deployment}. Rozdział opisuje
rozwiązania zastosowane w celu przetestowania systemu, zanim został uruchomiony
na prawdziwym dronie.

\section{Testy jednostkowe}
\section{Testy integracyjne}
\subsection{Symulacja i symulatory}
\section{Testy w terenie}


% \chapter{Podsumowanie}

W ramach projektu stworzony został system inspekcji obszarów, wykorzystujący autonomiczne drony.
W trakcie rozwoju system był regularnie testowany, zarówno na poziomie poszczególnych komponentów
jak i na poziomie całego systemu. Dzięki temu, w trakcie testów w terenie wszystkie
komponenty systemu bezkonfliktowo współgrały
ze sobą. Z punktu widzenia infrastruktury sieciowej,
testy w terenie nie różniły się niczym od testów wykonywanych na symulatorze --
dane telemetryczne odbierane z drona były dokładnie takie~same.

Osiągniętą dokładność rozpoznawania ludzi na zdjęciach wykonanych w trakcie lotu
można uznać za zadowalającą, zważywszy na fakt że praca nie skupiała się na
algorytmach rozpoznawania obrazu. Wykorzystanie otwartych zbiorów danych, zawierających
oznakowane zdjęcia wykonane z samolotów i dronów (przykładowo \textit{VisDrone Dataset} \cite{visdrone}),
może poprawnić dokładność rozpoznawania obiektów na zdjęciach.

Jak wspomniano w podrozdziale \ref{early_tests}, automatyczne aktualizowanie infrastruktury
internetowej systemu pozwoliło na szybsze i bezpieczniejsze wprowadzanie poprawek w systemie.
Dzięki temu udało się uniknąć potencjalnych błędów przy wdrażaniu nowych funkcjonalności.

Zastosowane w pracy rozwiązania kwalifikują się do metodyki \textit{DevOps}, opierającej
się na zacieśnieniu więzów pomiędzy programistami i administratorami systemu.
Automatyzacja procesów związanych z testami i wdrożeniami oraz powiązanie ich
z repozytorium projektowym zwiększa u programistów świadomość tego, jak ważny jest proces wdrażania.
W przypadku systemu wykorzystującego realne, fizyczne i drogie komponenty, pozwala to 
przyspieszyć ,,dojrzewanie'' systemu -- czas, po którym programiści mogą być pewni stabilności
działania produktu, nad którym pracują.

W początkowej fazie rozwoju projektu, dodatkowy wysiłek związany z budową infrastruktury
odpowiedzialnej za wdrożenia może wydawać się zupełnie zbędny. Nie jest to jednak 
błąd, jak w przypadku przedwczesnej optymalizacji. Wczesne zdefiniowanie ram projektu
pozwala na utrzymanie rozwoju kodu w ryzach. Jest niezbędne w celu utrzymania
stałego kursu ku wyznaczonym w projekcie celom.

% nk zdanie z agregacją entropii


% \bibliographystyle{plalpha}
% \bibliographystyle{plabbrv}
% \bibliographystyle{plain}
\bibliographystyle{ieeetr}

%UWAGA: bibliotekę referencji należy przygotować samemu. Dobrym do tego narzędziem jest JabRef.
%       Nazwę przygotowanej biblioteki wpisuje się poniżej bez rozszerzenia 
%       (w tym przypadku jest to "dokumentacja.bib")
\setlength{\bibitemsep}{2pt} % - by zacieśnić wykaz literatury
\bibliography{dokumentacja}
\appendix
% \chapter{Tytuł dodatku}
Zasady przyznawania stopnia naukowego doktora i doktora habilitowanego w Polsce określa ustawa z dnia 14 marca 2003 r. o stopniach naukowych i~tytule naukowym oraz o stopniach i~tytule w zakresie sztuki (Dz.U. nr 65 z 2003 r., poz. 595 (Dz. U. z 2003 r. Nr 65, poz. 595). Poprzednie polskie uregulowania nie wymagały bezwzględnie posiadania przez kandydata tytułu zawodowego magistra lub równorzędnego (choć zasada ta zazwyczaj była przestrzegana) i zdarzały się nadzwyczajne przypadki nadawania stopnia naukowego doktora osobom bez studiów wyższych, np. słynnemu matematykowi lwowskiemu – późniejszemu profesorowi Stefanowi Banachowi. 

W innych krajach również zazwyczaj do przyznania stopnia naukowego doktora potrzebny jest dyplom ukończenia uczelni wyższej, ale nie wszędzie.


% \chapter{Opis załączonej płyty CD/DVD}
Tutaj jest miejsce na zamieszczenie opisu zawartości załączonej płyty.
Należy wymienić, co zawiera.

\chapterstyle{noNumbered}
\phantomsection % sets an anchor
\addcontentsline{toc}{chapter}{Indeks rzeczowy}
\printindex

\end{document}
