\chapter{Przegląd i wybór technologii, architektura systemu}

\section{Zarys architektury}

\begin{tikzpicture}[auto,node distance=1.5cm]
	% Create an entity with ID node1, label "Fancy Node 1".
	% Default for children (ie. attributes) is to be a tree "growing up"
	% and having a distance of 3cm.
	%
	% 2 of these attributes do so, the 3rd's positioning is overridden.
	\node[entity] (backend) {Serwer webowy}
	  [grow=up,sibling distance=3cm];

	\node[entity](telem) {Podsystem obsługujący telemetrię}
		[below right = of backend];
	% Now place a relation (ID=rel1)
	% Now the 2nd entity (ID=rel2)
	\node[entity] (dron) [below left = of backend] {Dron};
	
	% \node[entity] (backend) [above right = of rel1]	{Serwer webowy};
	
	% Draw an edge between rel1 and dron; rel1 and node2
	% \path (backend) edge node {1-\(m\)} (dron);
\end{tikzpicture}

% \begin{tikzpicture}
	
% 	\node[entity](backend) {Serwer webowy}
% 	[];
% 	\node[entity](backend_telem) {Podsystem obsługujący telemetrię}[
% 		below right = of backend
% 	];

% 	% \node[entity] (dron) {Dron}
% 	%   [grow=up,sibling distance=3cm];
	
	
	 
% \end{tikzpicture}

\section{Oprogramowanie na dronie}

\subsection{Kontroler lotu}

Wielowirnikowce podłączone do systemu, muszą być
wyposażone w kontroler lotu - umożliwiający
autonomiczny lot, stabilizację oraz obsługę 
peryferiów takich jak czujniki oraz silniki.

Spośród aktywnie rozwijanych i popularnych
projektów \cite{autopilots_sourvey} tworzących oprogramowanie do kontrolerów
lotu, można wyróżnić cztery najpopularniejsze inicjatywy: \\

\begin{table}[htb]
	\centering\small
	\caption{
		Najpopularniejsze otwarte projekty
		oprogramowania obsługującego kontrolery lotu
	}
	\label{tab}

	\begin{tabularx}{0.87\textwidth}
	{ 
	| >{\raggedright\arraybackslash}l 
	| c 
	| >{\raggedright\arraybackslash}X
	| >{\raggedleft\arraybackslash}l |
	}
	\hline
	\textbf{Nazwa projektu} & \textbf{Rok założenia} &
	\textbf{Docelowy hardware}
	&  \textbf{Licencja}
	\\\hline
	ArduPilot\cite{ardupilot_home_page}		&  2009	& otwarte mikrokontrolery ARM & GPLv3
	\\ \hline
	AutoQuad\cite{autoquad_timeline}		&  2011	& mikrokontrolery STM Cortex M4		 & GPLv3
	\\ \hline
	LibrePilot\cite{librepilot_home_page}	&  2015	& zamknięte źródłowo kontrolery lotu, bazujące na architekturze ARM & GPLv3
	\\ \hline       
	PX4 Autopilot\cite{px4_home_page}		&  2012	& otwarte mikrokontrolery ARM & BSD
	\\ \hline       
	\end{tabularx}
	
\end{table}


\subsection{eee reszta?}

Poza kontrolerem lotu, który zawiera jedynie oprogramowanie 
niezbędnie do sterowania lotem i udostępniania
strumienia telemetrii (zazwyczaj przez port szeregowy),
na maszynie musi znaleźć się też drugi komputer - 
do zastosowań bardziej ogólnych.

\section{Protokoły wymiany danych}

\section{Oprogramowanie serwerowe}

\section{Oprogramowanie klienckie}

\section{Struktura repozytoriów}

% \section{Praca z wieloma repozytoriami}
\section{Wspólne punkty stykowe - \texttt{git submodules}}

% \chapter{Metodyka pracy i zarządzania}
% \section{Efektywne wykorzystanie narzędzi dostępnych w popularnych 
% systemach kontroli wersji}
