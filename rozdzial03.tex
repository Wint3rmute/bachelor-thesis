\chapter{Przegląd i wybór technologii, architektura systemu}

\section{Oprogramowanie na dronie}

\subsection{Kontroler lotu}

Wielowirnikowce podłączone do systemu, muszą być
wyposażone w kontroler lotu - umożliwiający
autonomiczny lot, stabilizację oraz obsługę 
peryferiów takich jak czujniki oraz silniki.

Spośród aktywnie rozwijanych i popularnych
projektów tworzących oprogramowanie do kontrolerów
lotu, można wyróżnić trzy najpopuparniejsze inicjatywy: \\

\begin{table}[htb]
	\centering\small
	\caption{Najpopularniejsze otwarte projekty oprogramowania obsługującego kontrolery lotu}
	\label{tab}

	\begin{tabular}{|l|l|l|l|}
		\hline 
		\textbf{Nazwa projektu}						& \textbf{Rok założenia} &  \textbf{Licencja} \\ \hline
		Ardupilot\cite{ardupilot_home_page}			&     2009				 & GPL     \\ \hline
		PX4 Autopilot\cite{px4_home_page}						&     2012				 & BSD    \\ \hline
		Paparazzi UAV\cite{paparazzi_home_page}		&     2005				 & GPL \\ \hline    
	\end{tabular}
\end{table}


\subsection{eee reszta?}

Poza kontrolerem lotu, który zawiera jedynie oprogramowanie 
niezbędnie do sterowania lotem i udostępniania
strumienia telemetrii (zazwyczaj przez port szeregowy),
na maszynie musi znaleźć się też drugi komputer - 
do zastosowań bardziej ogólnych.

\section{Protokoły wymiany danych}

\section{Oprogramowanie serwerowe}

\section{Oprogramowanie klienckie}

\section{Struktura repozytoriów}

% \section{Praca z wieloma repozytoriami}
\section{Wspólne punkty stykowe - \texttt{git submodules}}

% \chapter{Metodyka pracy i zarządzania}
% \section{Efektywne wykorzystanie narzędzi dostępnych w popularnych systemach kontroli wersji}
