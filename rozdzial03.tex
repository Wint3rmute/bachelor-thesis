\chapter{ Architektura systemu: przegląd i wybór technologii }

\section{Zarys architektury}

% \begin{tikzpicture}[auto,node distance=1.5cm]
% 	% Create an entity with ID node1, label "Fancy Node 1".
% 	% Default for children (ie. attributes) is to be a tree "growing up"
% 	% and having a distance of 3cm.
% 	%
% 	% 2 of these attributes do so, the 3rd's positioning is overridden.
% 	\node[entity] (backend) {\textbf{Serwer webowy}}
% 	  [grow=up,sibling distance=1cm];

% 	  \node[entity, node distance=0.5cm] (telem) [below  = of backend] {\footnotesize{Obsługa telemetrii}};
% 	  \node[entity, node distance=0.5cm, align=center] (api) [below right = of backend] {\footnotesize{Zapis tras i harmonogramu lotów}};
% 	  \node[entity, node distance=0.5cm] (api) [below left = of backend] {\footnotesize{Archiwizacja logów}};
% 	% Now place a relation (ID=rel1)
% 	% Now the 2nd entity (ID=rel2)
% 	\node[entity, node distance=3cm] (dron) [below left = of backend] {Dron};
% 	% \path (dron) edge node {n-\(1\)} (backend);

% 	\node[entity, node distance=2.5cm] (flight_controller) [ below left of = dron] {Kontroler lotu};
% 	\node[entity, node distance=2.5cm] (flight_computer) [ below right of = dron] {Komputer pokładowy};
% 	\node[entity, node distance=1.5cm] (camera) [ below of = flight_computer] {Kamera};

% 	\node[entity, node distance=3cm] (frontend) [below right = of backend] {Aplikacja kliencka};
% 	% \path (backend) edge node {1-\(n\)} (frontend);




% \end{tikzpicture}
% \begin{tikzpicture} [level distance=2cm] %[level 1/.style={sibling distance=2cm},level 2/.style={sibling distance=10mm}]


% 	\tikzstyle{level 1}=[sibling distance=4.5cm]
% 	\tikzstyle{level 2}=[sibling distance=3.5cm]
% 	\tikzstyle{level 3}=[sibling distance=2mm]
		
% 	\node[](root){}
% 		child {node [rectangle,draw] (dron) {Dron}
% 			edge from parent[draw=none]
% 			child {node [rectangle,draw](flight_controller){Kontroler lotu}}
% 			child {node [rectangle,draw](flight_computer){Komputer pokładowy}
% 				child {node [rectangle,draw, yshift=1.1cm](camera){Kamera}}
% 			}
% 		}
% 		child {node [rectangle,draw, yshift=1.5cm, xshift=-0.5cm] (backend) {Serwer webowy}
% 			edge from parent[draw=none]
% 			child {node [rectangle,draw](flight_controller){Rest API}}
% 			child {node [rectangle,draw, align=center](ai){Analiza\\zdjęć}}
% 			child {node [rectangle,draw, align=center](flight_controller){Obsługa\\telemetrii}}
% 		}
% 		child {node [rectangle,draw] (frontend) {Aplikacja kliencka}
% 			edge from parent[draw=none]
% 			child {node [rectangle,draw, align=center](flight_controller){Odbiór\\telemetrii}}
% 			child {node [rectangle,draw, align=center](flight_controller){Planowanie tras i \\ harmonogramów lotów}}
% 		};
		

% \end{tikzpicture}

\begin{figure}[H]
\centering\small
\caption{
	Zarys architektury systemu
}
\label{fig}
\hspace{-1.2cm}
\begin{forest}
	% forest preamble: determine layout and format of tree
	direction switch,
	for tree={fork sep=3em}
	[ System inspekcji obszarów,yshift=3em,alias=LP
	  [ \textbf{Dron}
		[ Kontroler lotu ]
		[ Komputer pokładowy
		 [ Wyznaczanie trasy lotu ]
		 [ Wysyłanie telemetrii ]
		 [ Wysyłanie zdjęć ]
		]
		[ Kamera ]
		[ Modem GSM ]
	  ]
	  [ \textbf{Serwer webowy}, yshift=0.5cm
		[ Obsługa telemetrii
			[ Odbiór danych z drona ]
			[ Przekazywanie do klientów ]
			[ Przygotowanie do archiwizacji ]
		]
		[ Rest API: odczyt i zapis
		  [ Logów telemetrii ]
		  [ Tras lotów ] 
		  [ Harmonogramu lotów ]
		  [ Raportów z lotów ]
		]
		[ Sztuczna inteligencja
		  [ Analiza obrazów z lotów ]
		]
	  ]
	  [ \textbf{Aplikacja kliencka}, xshift=0.1cm
		[ Planowanie 
			[ Tras lotów ]
			[ Harmonogramu lotów ]
		]
		[ Odbiór telemetrii ]
		[ Przegląd raportów z lotów 
		  [ Trasa lotu ]
		  [ Rozpoznane na zdjęciach obiekty ]
		]
	  ]
	]
%   \draw[thick]
%   ([yshift=-1.5em]LP.south)  -- ++(-6em,0) node[left,draw,font=\sffamily,thin]{ABC}
%   ([yshift=-1.5em]LP.south)  -- ++(6em,0) node[right,draw,font=\sffamily,thin]{XYZ};
  \end{forest}
\end{figure}

% \begin{tikzpicture}
	
% 	\node[entity](backend) {Serwer webowy}
% 	[];
% 	\node[entity](backend_telem) {Podsystem obsługujący telemetrię}[
% 		below right = of backend
% 	];

% 	% \node[entity] (dron) {Dron}
% 	%   [grow=up,sibling distance=3cm];
	
	
	 
% \end{tikzpicture}

\section{Dron}

\subsection{Kontroler lotu}

Wielowirnikowce podłączone do systemu, muszą być
wyposażone w kontroler lotu -- umożliwiający
autonomiczny lot, stabilizację oraz obsługę 
peryferiów takich jak czujniki oraz silniki.

Spośród aktywnie rozwijanych i popularnych
projektów \cite{autopilots_sourvey} tworzących oprogramowanie do kontrolerów
lotu, można wyróżnić cztery najpopularniejsze inicjatywy: 

\begin{table}[htb]
	\centering\small
	\caption{
		Najpopularniejsze otwarte projekty
		oprogramowania obsługującego kontrolery lotu
	}
	\label{tab}

	\begin{tabularx}{0.87\textwidth}
	{ 
	| >{\raggedright\arraybackslash}l 
	| c 
	| >{\raggedright\arraybackslash}X
	| >{\raggedleft\arraybackslash}l |
	}
	\hline
	\textbf{Nazwa projektu} & \textbf{Rok założenia} &
	\textbf{Docelowy hardware}
	&  \textbf{Licencja}
	\\\hline
	ArduPilot\cite{ardupilot_home_page}		&  2009	& otwarte mikrokontrolery ARM & GPLv3
	\\ \hline
	AutoQuad\cite{autoquad_timeline}		&  2011	& mikrokontrolery STM Cortex M4		 & GPLv3
	\\ \hline
	LibrePilot\cite{librepilot_home_page}	&  2015	& zamknięte źródłowo kontrolery lotu, bazujące na architekturze ARM & GPLv3
	\\ \hline       
	PX4 Autopilot\cite{px4_home_page}		&  2012	& otwarte mikrokontrolery ARM & BSD
	\\ \hline       
	\end{tabularx}
	
\end{table}

Spośród wymienionych projektów, ArduPilot posiada najbardziej rozbudowaną bazę 
dokumentacji i instrukcji. Architektura projektu umożliwia skompilowanie projektu
na standardowy komputer typu PC uruchomienie go w wirtualnym
środowisku\cite{ardupilot_sitl}, co ułatwia proces testowania oprogramowania,
które steruje dronem -- model maszyny jest symulowany, jednak warstwa komunikacji
jest dokładnie taka sama, jak w przypadku pracy z prawdziwym dronem. Dodatkowo, projekt
realizowany był w kole studenckim, w którym ArduPilot jest od lat wykorzystywany
do sterowania dronami i samolotami, co przesądziło o zastosowaniu go jako
oprogramowanie do kontrolera lotu.

\subsection{Komputer pokładowy}

Poza kontrolerem lotu, który zawiera jedynie oprogramowanie 
niezbędnie do sterowania dronem i udostępniania
strumienia telemetriii, na maszynie musi znaleźć się też komputer pokładowy, 
do zastosowań bardziej ogólnych. Będzie on wykorzystywany do
obsługi wysokopoziomowych peryferiów: kamery oraz modemu GSM. Do zadań
komputera pokładowego będzie należała także realizacja logiki biznesowej systemu
- odczytanie harmonogramu przelotów z serwera webowego i załadowanie do kontrolera lotu
konkretnej trasy przelotu.

\subsubsection{Wymagania sprzętowe}

Aby wypełniać zadania wymienione powyżej, komputer pokładowy musi
posiadać następujące interfejsy sprzętowe:

\begin{itemize}
	\item \texttt{UART} -- do komunikacji z kontrolerem lotu,
	\item \texttt{USB} -- do komunikacji z modemem,
	\item \texttt{CSI} -- do komunikacji z kamerą.
\end{itemize}

Większość dostępnych na rynku komputerów klasy SBC (\textit{Single-board Computer})
posiada powyższe interfejsy, więc wymagania sprzętowe nie są tutaj ograniczeniem.
W projekcie został wykorzystany najpopularniejszy do zastosowań amatorskich
komputer \textit{Raspberry Pi}. 

\subsubsection{Oprogramowanie}

Na komputerze pokładowym zainstalowany jest system Linux -- gwarantuje to bezproblemową
obsługę peryferiów oraz dostępność stosu sieciowego, koniecznego do przesyłania 
danych telemetrycznych i zdjęć.
Dodatkowo, wymagane jest oprogramowanie dekodujące
i enkodujące wiadomości protokołu \textit{MavLink}, który jest wykorzystywany przez 
ArduPilota do komunikacji z zewnętrznymi systemami \cite{ardupilot_mavlink}.

Infrastruktura projektu ArduPilot dostarcza gotowe narzędziae do
parsowania i tworzenia wiadomości w protokole \textit{MavLink}.
Jedną z nich jest \texttt{pymavlink} - implementacja protokołu \textit{MavLink} 
w języku Python. \texttt{pymavlink} zawiera podstawowe funkcje i obiekty konieczne
do komunikacji z kontrolerem lotu. Biblioteka jest niskopoziomowa i nie w całości
napisana w sposób obiektowy -- finalnie wykorzystujemy więc bibliotekę
\texttt{dronekit-python}\cite{dronekit_python}, która rozbudowuje \texttt{pymavlink}
o w pełni obiektowy, wysokopoziomowy interfejs do komunikacji z kontrolerem lotu. 
Skrypty odpowiedzialne za logikę biznesową napisane są w Pythonie.

\section{Protokoły wymiany danych}

\subsection{Dane telemetryczne - biblioteka \texttt{protobuf}}

W trakcie lotu, drony regularnie wysyłają dane telemetryczne, identyfikując się,
podając number lotu oraz informują o swoim obecnym położeniu a także podają stan baterii.
Nie jest wykluczone, że w przyszłości może pojawić się potrzeba dołączenia do danych
telemetrycznych nowych informacji, na przykład w przypadku, gdy do maszyny zostanie
dodane nowe oprzyrządowanie, lub gdy system byłby adaptowany do obsługi innego rodzaju misji.

Dodatkowo, skala projektu wymusza utrzymywanie implementacji tego samego protokołu w wielu
różnych językach programowania -- oprogramowanie wysyłające telemetrię z drona napisane
jest w Pythonie, interfejs webowy wyświetlający telemetrię będzie musiał być jednak napisany
w języku JavaScript (ponieważ tylko ten język jest wspierany przez przeglądarki internetowe).
W przyszłości może pojawić się konieczność dodania wsparcia dla innego języka programowania,
niestety wraz z dodawaniem coraz to kolejnych danych telemetrycznych i zwiększaniem liczby
wspieranych języków, rośnie szansa na popełnienie błędu w implementacji protokołu.

Jednym z możliwych rozwiązań tego problemu jest wykorzystanie narzędzi do generowania
kodu. Narzędzia tego typu pozwalają na zdefiniowanie standardu danych telemetrycznych
we własnym (specyficznym dla narzędzia) języku, opisującym strukturę danych przesyłanych
przez protokół. Następnie, narzędzia takie generują implementację protokołu w wybranych
językach programowania. Dzięki automatycznemu generowaniu implementacji protokołu,
programiści nie muszą dbać o spójność implementacji pomiędzy wieloma językami.

Wykorzystywanym w projekcie narzędziem do generowania implementacji protokołu jest
\texttt{protobuf} (\textit{Protocol Buffers})\cite{protocol_buffers} -- biblioteka
napisana w firmie Google, stworzona z myślą o zapewnianiu wydajnej komunikacji w czasie
rzeczywistym pomiędzy wieloma systemami informatycznymi. W obecnej wersji
(\texttt{v3.14.0}), biblioteka pozwala na generowanie kodu w językach:

\begin{multicols}{2}
\begin{itemize}
	\item Python
	\item C++
	\item JavaScript
	\item Go
	\item Java
	\item C\#
\end{itemize}
\end{multicols}

\begin{figure}[H]
	\centering\small
	\caption{
	 Diagram przedstawiający system generowania implementacji protokołu, na podstawie
	 narzędzia \texttt{protobuf}	
	}
	\label{fig}
	\hspace{-1.2cm}
\tikzstyle{arrow} = [thick,->,>=stealth]
\begin{tikzpicture}[node distance=5cm]
	
	\node
		[entity](definitions)
		{ Definicje wiadomości };
	
	\node
		[trapezium, trapezium left angle=70, trapezium right angle=110, draw=black, right of = definitions, align=center](coompiler)
		{ Kompilator \texttt{protobuf} };
	
	\draw [arrow] (definitions) -- (coompiler);	
	
	% \node
	% 	[align=center, right of = coompiler, xshift=1.5cm](implementations)
	% 	{ Implementacje w wybranych \\ językach programowania };

	% \draw [arrow] (coompiler) -- (8.5, 0.4);	
	% \draw [arrow] (coompiler) -- (8.6, 0);	
	% \draw [arrow] (coompiler) -- (8.5, -0.4);
			
	\node
		[align = center, right of = coompiler, yshift=1.2cm, xshift=1.5cm](implementation_a)
		{Implementacja w języku A};
	\draw [arrow] (coompiler) -- (implementation_a.west);	
	
	\node
		[align = center, right of = coompiler, xshift=1.5cm](implementation_b)
		{Implementacja w języku B};
	\draw [arrow] (coompiler) -- (implementation_b);
	
	\node
		[align = center, right of = coompiler, yshift=-1.2cm, xshift=1.5cm](implementation_c)
		{Implementacja w języku C};	
	\draw [arrow] (coompiler) -- (implementation_c.west);

\end{tikzpicture}
\end{figure}

Poza automatycznym generowaniem
kodu protokołu, \texttt{protobuf} zapewnia także:

\begin{itemize}
	\item wydajne wykorzystanie miejsca -- biblioteka upakowuje dane w formie binarnej,
	\item kompatybilność wsteczną -- w przypadku dodania nowego typu pakietu telemetrii,
	implementacja zachowuje spójność z poprzednimi wersjami. Dzięki temu możliwe jest
	stopniowe wprowadzanie zmian w wielokomponentowym systemie,
	\item automatycznie generowane mechanizmy, pozwalające na sprawdzanie, czy dane 
	umieszczone w pakiecie telemetrii są poprawne (sprawdzanie typów, sprawdzanie
	czy zostały wypełnione wszystkie pola pakietu),
	\item interfejs programistyczny, umożliwiający samodzielne dopisanie wsparcia
	nowych języków do kompilatora \texttt{protobuf}.
\end{itemize}

\begin{lstlisting}[label=list:protobuf,caption=Przykład definicji pakietu \texttt{protobuf}, basicstyle=\footnotesize\ttfamily]
syntax = "proto3";

package Telemetry;

/**
  Wiadomość zawierająca pozycję
  maszyny w trakcie lotu
*/
message Position {
  float lat = 1;
  float lng = 2;
  float alt = 3;
  float heading = 4;
}

message TelemFrameHeader {
  /* Flight metadata */
  fixed32 timestamp = 1;
  int32 machine_id = 2;
  fixed32 flight_id = 3;

  /**
	Kompozycja wcześniej
	zdefiniowanej wiadomości 
  */
  Position position = 4;
}
\end{lstlisting}

\begin{lstlisting}[label=list:protobuf,caption=Przykład wykorzystania wygenerowanej implementacji pakietów w języku Python,basicstyle=\footnotesize\ttfamily]
# Wygenerowany moduł zawierający definicje
# pakietów, domyślnie nosi nazwę definitions_pb2
import definitions_pb2
import time

header = definitions_pb2.TelemFrameHeader()

header.timestamp = int(time.time())
header.machine_id = 1
header.flight_id = 5

header.position.lat = 5
header.position.lng = 10
header.position.alt = 15
header.position.heading = 20

# Postać binarna, gotowa do zapisania do pliku
# lub wysłania przez protokół UDP/WebSocket
serialised_header = header.SerializeToString()
print("Serialised header:")
print(serialised_header) 
\end{lstlisting}

\subsection{Zdjęcia wysyłane w trakcie lotu - biblioteka \texttt{imagezmq}}

Wykonane przez drony zdjęcia wysyłane są na serwer webowyza pomocą biblioteki
\texttt{imagezmq}. Jak zaznaczono we wstępie, wysyłanie strumienia wideo to
temat zbyt skomplikowany, aby poruszać go w pracy -- rozwiązanie wykorzystywane
do przesyłu zdjęć zostało wybrane ze względu na prostotę instalacji i implementacji.

Biblioteka \texttt{imagezmq} pozwala wysyłać obrazy za pomocą protokołu \texttt{zmq}.
Przed wysłaniem, obrazy są kompresowane, jednak jest to kompresja ograniczająca się do 
pojedynczego kadru -- nie zaś strumienia obrazów, jak by to miało miejsce w przypadku 
strumieniowania filmu z lotu.

\section{Oprogramowanie serwerowe}

Serwer webowy jest punktem stykowym wszystkich elementów w systemie.
Na serwerze przechowywane są wszystkie dane konieczne do działania systemu, takie jak
trasy i harmonogramy lotów. Nadchodząca z dronów telemetria jest na 
serwerze multipleksowana i archiwizowana, pozwalając zarówno na odbieranie jej
w czasie rzeczywistym z poziomu aplikacji klienckiej, jak i na analizę już zakończonych lotów.

\subsection{Obsługa telemetrii}

System obsługujący dane telemetryczne odbiera nadchodzące z maszyn pakiety
\texttt{protobuf}, wysyłane w ramkach UDP, następnie przekazuje je do 
podłączonych przez WebSockety aplikacji klienckich - w ten sposób możliwy jest
jednoczesny odbiór telemetrii na dowolnie wielu urządzeniach. Pojedyncze loty rozróżnialne
są za pomocą unikatowej krotki \texttt{(id\_maszyny, id\_lotu)}. Gdy system zauważy,
że pakiety telemetryczne z danego lotu przestały napływać (przez ponad 10 sekund
nie pojawił się żaden nowy pakiet o danym \texttt{(id\_maszyny, id\_lotu)}), zapisuje
wszystkie otrzymane w ramach tego lotu pakiety.


\begin{figure}[H]
	\centering\small
	\caption{
	 Architektura systemu obsługującego telemetrię 
	}
	\label{fig}
	\hspace{-1.2cm}
\tikzstyle{arrow} = [thick,->,>=stealth]
\begin{tikzpicture}[node distance=8cm]
	
	\node
		[circle, draw=black](drone)
		{ Dron };
	
	\node
		[entity, right of = drone, align=center, xshift=-1.5cm](telemetry_server)
		{ Serwer telemetrii };
	
	\draw [arrow] (drone) -- (telemetry_server) node[pos=0.5, below]{Ramki UDP};	
	
				
	\node
		[entity, align = center, right of = coompiler, yshift=1cm](client_1)
		{ Aplikacja kliencka };
	\draw [arrow] (telemetry_server) -- (client_1.west) node[pos=0.5, above, yshift=0.12cm]{WebSocket};	

	
	\node
		[entity, align = center, right of = coompiler, yshift=-1cm](client_2)
		{ Aplikacja kliencka };
	\draw [arrow] (telemetry_server) -- (client_2.west)node[pos=0.5, below, yshift=-0.09cm]{WebSocket};	


\end{tikzpicture}
\end{figure}


\subsection{Rest API}

\subsection{Sztuczna inteligencja - analiza obrazów z lotów}


\section{Oprogramowanie klienckie}

\section{Struktura repozytoriów}

% \section{Praca z wieloma repozytoriami}
\section{Wspólne punkty stykowe - \texttt{git submodules}}


\section{Podsumowanie architektury}

% \chapter{Metodyka pracy i zarządzania}
% \section{Efektywne wykorzystanie narzędzi dostępnych w popularnych 
% systemach kontroli wersji}
