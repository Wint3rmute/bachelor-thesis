\chapter{Wymagania funkcjonalne systemu}

\section{Oprogramowanie na dronie}
\section{Protokoły wymiany danych}
\section{Oprogramowanie serwerowe}
\section{Oprogramowanie klienckie}


% ================================== % 
\chapter{Wybór technologii i architektura systemu}

\section{Oprogramowanie na dronie}
\section{Protokoły wymiany danych}
\section{Oprogramowanie serwerowe}
\section{Oprogramowanie klienckie}

\section{Struktura repozytoriów}
\section{Praca z wieloma repozytoriami}
\section{Wspólne punkty stykowe - \texttt{git submodules}}

% \chapter{Metodyka pracy i zarządzania}
% \section{Efektywne wykorzystanie narzędzi dostępnych w popularnych systemach kontroli wersji}

% ================================== % 
\chapter{Wdrażanie systemu}

\section{Konteneryzacja}
\section{Automatyczne budowanie projektów}
\section{Automatyczne aktualizacje kontenerów}
\section{Automatyczne wdrażanie statycznego kodu}


% ================================== % 
\chapter{Testy systemu}

\section{Testy jednostkowe}
\section{Testy integracyjne}
\subsection{Symulacja i symulatory}
\section{Systemy ciągłej integracji}
\section{Testy w terenie}

% ================================== % 
\chapter{Podsumowanie}

\section{Wyniki testów}
\section{Osiągnięta sprawność}
\section{Pola do poprawy}
\section{Wnioski}

%%%2. środowisko do pisania kodu latexa: 
%%%( )
%%%3. viewer pdf-ów, pozwalający na nawigację zwrotną: Sumatra PDF 3.0
%%%(http://www.sumatrapdfreader.org/download-free-pdf-viewer.html)
%%%
%%%- o konfiguracji texniccenter do współdziałania z sumatra pdf można poczytać sobie na stronie:
%%%http://tex.stackexchange.com/questions/116981/how-to-configure-texniccenter-2-0-with-sumatra-2013-2014-2015-version
%%%(można znaleźć też inne tutoriale)
%%%
%%%4. środowisko do zarządzania bibliografią: JabRef
%%%(http://jabref.sourceforge.net/download.php)
%%%
%%%Polecam też instalację pod windowsami następujących narzędzi:
%%%- Sumatra PDF - przeglądarka pdf umożliwiająca nawigację pomiędzy
%%%edytowanym tekstem a przeglądanym dokumentem (podglądanie tekstu w
%%%TeXnicCenter umieszcza kursor w odpowiednim miejscu w pdfie, podwójne
%%%kliknięcie w pdfie ustawia kursor w edytorze tekstu).
%%%- JabRef - narzędzie do przygotowywania bibliografii.
%%%
%%%
%%%Uwaga: tytuł powinien zmieścić się w okienku kolorowej okładki (którą
%%%powinna dostarczyć uczelniana administracja). Proszę posterować
%%%parametrami, aby "wpasować" w okienko własny tekst.
%%%
%%%Do ASAPa należy wprowadzić pracę dyplomową/projekt inżynierski w pliku o nazwie:
%%%
%%%W04_[nr albumu]_[rok kalendarzowy]_[rodzaj pracy] (szczegółowa instrukcja pod adresem asap.pwr.edu.pl)
%%%
           %%%Przykładowo:
        %%%­W04_123456_2015_praca inżynierska.pdf     - praca dyplomowa inżynierska
        %%%W04_123456_2015_projekt inżynierski.pdf   - projekt inżynierski
        %%%W04_123456_2015_praca magisterska.pdf  - praca dyplomowa magisterska
%%%
              %%%rok kalendarzowy ? rok realizacji kursu „Praca dyplomowa” (nie rok obrony) 