\chapter{Praca z szablonem}
\section{Organizacja szablonu}
Szablon składa się z pliku głównego, plików z kodem kolejnych rozdziałów i dodatków (włączanych do kompilacji w dokumencie głównym), katalogów z plikami grafik (włączanymi do rysunków w rozdziałach), pliku ze skrótami (opcjonalny), pliku z danymi bibliograficznymi (plik \texttt{dokumentacja.bib}). Taki ,,układ'' zapewnia porządek oraz pozwala na selektywną kompilację rozdziałów. Wyjaśniając to dokładniej, podczas tworzenia szablonu przyjęto następującą konwencję:
\begin{itemize}
\item Plikiem głównym jest plik \texttt{Dyplom.tex}. To w nim znajdują się deklaracje wszystkich używanych styli, definicje makr oraz ustawień, jak również polecenie \verb+\begin{document}+.
\item Teksty rozdziałów są redagowane w osobnych plikach o nazwach zawierających numer rozdziału. Pliki te zamieszczone są w katalogu głównym (tym samym, co plik \texttt{Dyplom.tex}). I~tak \texttt{rozdzial01.tex} to plik pierwszego rozdziału (ze Wstępem), \texttt{rozdzial02.tex} to plik z treścią drugiego rozdziału itd. 
\item Teksty dodatków są redagowany w osobnych plikach o nazwach zawierających literę dodatku. Pliki te, podobnie do plików z tekstem rozdziałów, zamieszczane są w katalogu głównym. I~tak \texttt{dodatekA.tex} oraz \texttt{dodatekB.tex} to, odpowiednio, pliki z treścią dodatku A oraz dodatku B.
\item Pewnym wyjątkiem od reguły nazewniczej w przypadku plików z tekstem rozdziałów i dodatków jest plik \texttt{skroty.tex}. Jest to plik, w którym zamieszczono wykaz użytych skrótów. W jego nazwie nie występuje żaden numer czy porządkowa litera. 
\item Każdemu rozdziałowi i dodatkowi towarzyszy katalog przeznaczony do składowania dołączanych w nim grafik. I tak \texttt{rys01} to katalog na pliki z grafikami dołączanymi do rozdziału pierwszego, \texttt{rys02} to katalog na pliki z grafikami dołączanymi do rozdziału drugiego itd.
Podobnie \texttt{rysA} to katalog na pliki z grafikami dołączanymi w dodatku A itd.
\item W katalogu głównym zamieszczany jest plik \texttt{dokumentacja.bib} zawierający bazę danych bibliograficznych.
\end{itemize}

\begin{table}[htb]
\centering\small
\caption{Pliki źródłowe szablonu oraz wyniki kompilacji}
\label{tab:szablon}
\begin{tabularx}{\linewidth}{|p{.55\linewidth}|X|}\hline
Źródła & Wyniki kompilacji \\ \hline\hline
\verb?Dokument.tex? - dokument główny\newline
\verb?Dokument.tcp? -- szablon projektu \texttt{MiKTeX}\newline
\verb?rozdzial01.tex? -- plik rozdziału \texttt{01}\newline
\verb?...?\newline
\verb?dodatekA.tex? -- plik dodatku \texttt{A}\newline
\verb?...?\newline
\verb?rys01? -- katalog na rysunki do rozdziału \texttt{01}\newline
\verb?   |- fig01.png? -- plik grafiki\newline
\verb?   |- ...?\newline
\verb?...?\newline
\verb?rysA? -- katalog na rysunki do dodatku \texttt{A}\newline
\verb?   |- fig01.png? -- plik grafiki\newline
\verb?   |- ...?\newline
\verb?...?\newline
\verb?dokumentacja.bib? -- plik danych bibliograficznych\newline
\verb?Dyplom.ist? -- plik ze stylem indeksu\newline
\verb?by-nc-sa.png? -- plik z ikonami CC\newline
 &
\verb?Dyplom.bbl?\newline
\verb?Dyplom.blg?\newline
\verb?Dyplom.ind?\newline
\verb?Dyplom.idx?\newline
\verb?Dyplom.lof?\newline
\verb?Dyplom.log?\newline
\verb?Dyplom.lot?\newline
\verb?Dyplom.out?\newline
\verb?Dyplom.pdf? -- dokument wynikowy\newline
\verb?Dyplom.syntex?\newline
\verb?Dyplom.toc?\newline
\verb?Dyplom.tps?\newline
\verb?*.aux?\newline 
\verb?Dyplom.synctex?\newline\\
\hline
\end{tabularx}
\end{table}

Szablon przygotowano w systemie Windows stosując kodowanie \texttt{cp1250}. Można go wykorzystać również w innych systemach i przy innych kodowaniach. Jednakże wtedy konieczna jest korekta dokumentu \texttt{Dyplom.tex} odpowiednio do wybranego przypadku. Korekta ta polegać może na zamianie polecenia \verb+\usepackage[cp1250]{inputenc}+  na polecenie \verb+\usepackage[utf8]{inputenc}+ oraz konwersji kodowania istniejących plików ze źródłem latexowego kodu (plików o rozszerzeniu \texttt{*.tex} oraz \texttt{*.bib}).

Samo kodowanie plików może być źródłem paru problemów. Chodzi o to, że użytkownicy pracujący z edytorami tekstów pod linuxem mogą generować pliki zakodowane w UTF-8 bez BOM lub z BOM, a pod windowsem -- pliki z kodowaniem znaków \texttt{cp1250} zakodowanych w~ANSI. A z takimi plikami różne edytory różnie sobie radzą (w szczególności edytor TeXnicCenter czasami z niewiadomego powodu traktuje zawartość pliku jako UTF8 lub ANSI -- chyba sprawdza, czy w bufore nie ma jakichś znaków specjalnych i na tej podstawie interpretuje kodowanie). Bywa, że choć wszystko wygląda OK to jednak kompilacja latexowa ,,nie idzie''. Problemem mogą być właśnie pierwsze bajty, których nie widać w edytorze. 

Kodowanie znaków jest istotne również przy edytowaniu bazy danych bibliograficznych (pliku \texttt{dokumentacja.bib}). Aby \texttt{bibtex} poprawnie interpretował polskie znaki plik \texttt{dokumentacja.bib} powinien być zakodowany w ANSI, CR+LF (dla ustawień jak w szablonie). Do konwersji kodowania można użyć Notepad++ (jest tam opcja ,,konwertuj'' - nie mylić z opcją ,,koduj'', która przekodowuje znaki, jednak nie zmienia sposobu kodowania pliku).



\section{Kompilacja szablonu}
Kompilację szablonu może uruchamić na killka różnych sposobów. Wszystko zależy od używanego systemu operacyjnego, zaintalowanej na nim dystrybucji latexa oraz dostępnych narzędzi. Zazwyczaj kompilację rozpoczyna się wydając polecenie z linii komend lub uruchamia się ją za pomocą narzędzi zintegrowanych środowisk.

Kompilacja z linii komend polega na uruchomieniu w katalogu, w którym rozpakowano źródła szablonu, następującego polecenia:
\begin{lstlisting}[basicstyle=\ttfamily]
> pdflatex Dyplom.tex
\end{lstlisting}
gdzie \texttt{pdflatex} to nazwa kompilatora, zaś \texttt{Dyplom.tex} to nazwa głównego pliku redagowanej pracy. 
W przypadku korzystania ze środowiska \texttt{TeXnicCenter} należy otworzyć dostarczony w szablonie plik projektu \texttt{Dyplom.tcp}, a następnie uruchomić kompilację narzędziami dostępnymi w pasku narzędziowym.

Aby poprawnie wygenerowały się wszystkie referencje (spis treści, odwołania do tabel, rysunków, pozycji literaturowych, równań itd.) kompilację \texttt{pdflatex} należy wykonać dwukrotnie, a~czasem nawet trzykrotnie, gdy wygenerowane mają zostać odwołania do pozycji literaturowych oraz wykazu literatury. 

Wygenerowanie danych bibliograficznych zapewnia kompilacja \texttt{bibtex} uruchamiana po kompilacji \texttt{pdfltex}. Można to zrobić z linii komend:
\begin{lstlisting}[basicstyle=\ttfamily]
> bibtex Dyplom
\end{lstlisting}
lub wybierając odpowiednią pozycję z paska narzędziowego wykorzystywanego środowiska. Po kompilacji \texttt{bibtex} na dysku pojawi się plik \texttt{Dyplom.bbl}. Dopiero po kolejnych dwóch kompilacjach \texttt{pdflatex} dane z tego pliku pojawią się w wygenerowanym dokumencie. Podsumowując, po każdym wstawieniu nowego cytowania w kodzie dokumentu uzyskanie poprawnego formatowania dokumentu wynikowego wymaga powtórzenia następującej sekwencji kroków kompilacji:
\begin{lstlisting}[basicstyle=\ttfamily]
> pdflatex Document.tex
> bibtex Document
> latex Document.tex
> latex Document.tex
\end{lstlisting}
Szczegóły dotyczące przygotowania danych bibliograficznych oraz zastosowania cytowań przedstawiono w podrozdziale \ref{sec:literatura}.

W głównym pliku zamieszczono polecenia pozwalające sterować procesem kompilacji poprzez włączanie bądź wyłączanie kodu źródłowego poszczególnych rozdziałów. Włączanie kodu do kompilacji zapewniają instrukcje \verb+\include+ oraz \verb+\includeonly+. Pierwsza z nich pozwala włączyć do kompilacji kod wskazanego pliku (np.\ kodu źródłowego pierwszego rozdziału \verb+\chapter{Wstęp}


\section{Geneza pracy} \label{intro_genesis}
% kilka akapitów wprowadzających,
% z czego powinno wynikać, że zrobienie
% takiego systemu jest potrzebne i ma sens.

Jako członek Akademickiego Klubu Lotniczego - koła naukowego Politechniki Wrocławskiej,
zajmuję się oprogramowywaniem autonomicznych dronów i samolotów. W ramach realizacji
projektu z Wrocławskiego Funduszu Aktywności Studenckiej powstał, przedstawiony w
tej pracy, prototyp systemu inspekcji obszarów.

Sektor gospodarki związany z inżynierią bezzałogowych maszyn lotniczych znajduje się
w stanie dynamicznego rozwoju i generuje coraz większe przychody.
Tym samym przyciąga inwestorów skłonnych zainwestować w innowacyjne pomysły zarówno 
dużych przedsiębiorstw, jak i młodych konstruktorów. Międzynarodowe zawody sponsorowane
przez podmioty prywatne i organizacje rządowe umożliwiają pasjonatom wdrożenie w życie
ich wizji. Przykład stanowi \textit{UAV Challenge} - zawody skoncentrowane na
autonomicznych systemach wsparcia służb medycznych, współorganizowane przez rząd
australijskiego stanu Queensland, które przyciągają corocznie najlepsze zespoły z całego
świata\cite{uav_sponsors}. Potencjał inwestycyjny i naukowy, którym dysponuje ta branża,
sprzyja powstawaniu różnorodnych rozwiązań. W konsekwencji bezpośrednio niezwiązane z
lotnictwem sektory korzystają z dorobku autonomicznej awiacji. Tytułem przykładu: 
generowanie trójwymiarowych map terenu, przy użyciu zdjęć wykonanych z dronów znajduje
zastostowanie w dziedzinach takich jak górnictwo, geodezja czy militaria
\cite{uav_photogrametry}, natomiast dostarczanie towarów indywidualnym klientom z pomocą
dronów jest w stanie zrewolucjonizować przemysł transportowy.

Przeprowadzenie autonomicznej misji bezzałogową maszyną możliwe jest dzięki osiągnięciom
inżynierii oprogramowania. Entuzjaści programowania regularnie udoskonalają i rozbudowują
projekty open-source. Powszechna dostępność i możliwość korzystania ze sprawdzonych
i stabilnych rozwiązań zdecydowanie usprawnia proces realizowania
kreatywnych i innowacyjnych projektów. 


  

% Autonomiczne lotnictwo to dynamicznie rozwijający się sektor branży lotniczej.
% Technologie pozwalające na wykorzystanie autonomicznych dronów i samolotów
% w nowych projektach biznesowych są powszechnie dostępne. Istnieją
% zarówno systemy zamknięte, w pełni komercyjne, jak i projekty otwartoźródłowe,
% pozwalające na zapoznanie się z kodem źródłowym oprogramowania sterującego statkami
% powietrznymi i interakcję z aktywną społecznością pasjonatów, wspólnie rozwijającą
% projekt. 

% W świecie biznesu powstają coraz to nowe rozwiązania, wykorzystujące autonomiczne
% maszyny do świadczenia usług - od razu nasuwającym się rozwiązaniem
% jest autonomiczne dostarczanie paczek \cite{prime_air}, ale istnieją też znacznie
% bardziej ambitne projekty: dostarczanie defibrylatora wprost do domu osoby, u której
% wystąpiło zatrzymanie akcji serca \cite{10.1001/jama.2017.3957} lub generowanie
% trójwymiarowych map terenu, wykorzystując zdjęcia wykonanych z dronów
% \cite{uav_photogrametry}. Biznes jest stosunkowo młody, więc branża jest otwarta
% na innowatorów -- firmy takie jak Boeing i Lockheed Martin sponsorują międzynarodowe
% konkursy przeznaczone dla młodych konstruktorów \cite{sae_2018}. 

% Zainteresowani autonomicznym lotnictwem inwestorzy nie ograniczają się do
% prywatnych firm. Rząd australijskiego stanu Queensland współorganizuje
% \textit{UAV Challenge} -- zawody skupione wokół rozwijania systemów wspierających 
% służby medyczne \cite{uav_sponsors}.

% Wykorzystanie otwartych technologii, skupionych wokół awiacji autonomicznej
% i połączenie ich z nowoczesnymi praktykami wdrażania oprogramowania to temat atrakcyjny
% zarówno z perspektywy inżynierii oprogramowania, jak i z perspektywy biznesowej. 
% % zapytać o to
% Koło naukowe Politechniki Wrocławskiej -- \textit{Akademicki Klub Lotniczy} (AKL) zajmuje
% się budowaniem i rozwijaniem autonomicznych dronów i samolotów \cite{akl_home_page}.
% Omawiany w pracy system powstał w ramach realizacji projektu
% z Wrocławskiego Funduszu Aktywności Studenckiej\cite{fast_webpage}.




% Szczególnie interesujące są zagadnienia integracji komponentów systemu,
% oraz testowanie - które w przypadku systemu angażującego rzeczywiste
% maszyny nie może ograniczyć się jedynie do standardowych testów jednostkowych.

\newpage
\section{Cel pracy} \label{intro_objective}
% koniecznie sformułowania:
% - Celem ogólnym pracy jest…?,
% - Celami szczegółowymi pracy są…?

Celem pracy jest stworzenie prototypu systemu inspekcji terenów, wykorzystującego
autonomiczne drony. System ma integrować się z już istniejącym oprogramowaniem
sterującym autonomicznymi maszynami, oraz wykorzystywać napisaną na potrzeby pracy
infrastrukturę informatyczną, pozwalającą na planowanie tras lotów, obsługę telemetrii
i rozpoznawanie obiektów na zdjęciach wykonanych w czasie lotu, za pomocą algorytmów
sztucznej inteligencji. Finalnie, system ma generować raport podsumowujący każdy lot,
w którym zawarta będzie trasa pokonana przez drona i zdjęcia wykonane w trakcie lotu,
wraz z rozpoznanymi na nich obiektami. 

Architektura systemu musi pozwalać na zautomatyzowanie procesu wdrażania
systemu, oraz zautomatyzowanie wdrażania nowych funkcjonalności - każde
z wdrożeń musi być poprzedzone testami integracyjnymi na poziomie całego systemu. 

Prototyp ma być w pełni testowalny, zarówno na poziomie pojedynczych
elementów systemu jak i na poziomie integracji całego projektu - testy muszą
angażować wszystkie komponenty systemu, uruchomione wewnątrz w pełni
zautomatyzowanego środowiska testowego.

\section{Zakres pracy} \label{intro_scope}

Zakres pracy obejmuje elementy projektu związane z
inżynierią i architekturą oprogramowania - proces projektowania struktury systemu,
wybór technologii, zaprojektowanie punktów stykowych w systemie, automatyzacja
procesu wdrażania systemu i nowych funkcjonalności.

Praca opisuje także sposób testowania systemu - od weryfikacji poprawności działania
poszczególnych komponentów, po pełne automatyczne testy integracyjne, wykorzystujące 
wszystkie komponenty systemu wraz ze zintegrowanym symulatorem drona. 
+). Druga, jeśli zostanie zastosowana, pozwala określić, które z~plików zostaną skompilowane w całości (na przykład kod źródłowy pierwszego i drugiego rozdziału \verb+\includeonly{rozdzial01.tex,rozdzial02.tex}+).
Brak nazwy pliku na liście w poleceniu \verb+\includeonly+ przy jednoczesnym wystąpieniu jego nazwy w poleceniu \verb+\include+ oznacza, że w kompilacji zostaną uwzględnione referencje wygenerowane dla tego pliku wcześniej, sam zaś kod źródłowy pliku nie będzie kompilowany. 

W szablonie wykorzystano klasę dokumentu \texttt{memoir} oraz wybrane pakiety. Podczas kompilacji szablonu w \texttt{MikTeXu} wszelkie potrzebne pakiety zostaną zainstalowane automatycznie (jeśli \texttt{MikTeX} zainstalowano z opcją dynamicznej instalacji brakujących pakietów). W przypadku innych dystrybucji latexowych może okazać się, że pakiety te trzeba doinstalować ręcznie (np.\ pod linuxem z \texttt{TeXLive} trzeba doinstalować dodatkową zbiorczą paczkę, a jeśli ma się menadżera pakietów latexowych - to pakiety latexowe można instalować indywidualnie).

Jeśli w szablonie będzie wykorzystany indeks rzeczowy, kompilację źródeł trzeba będzie rozszerzyć o kroki potrzebne na wygenerowanie plików pośrednich \texttt{Dokument.idx} oraz \texttt{Dokument.ind} oraz dołączenia ich do finalnego dokumentu (podobnie jak to ma miejsce przy generowaniu wykazu literatury).
Szczegóły dotyczące generowania indeksu rzeczowego opisano w podrozdziale~\ref{sec:indeks}.