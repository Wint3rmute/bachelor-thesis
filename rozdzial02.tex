\chapter{Wymagania funkcjonalne systemu}

\section{Oprogramowanie na dronie}

Wielowirnikowce podłączone do systemu muszą być
zdolne do autonomicznego lotu - w kontekście pracy
oznacza to zdolność do automatycznego startu,
lądowania, stabilizacji oraz samodzielnego
lotu do koordynatów GPS. W trakcie lotu, maszyna musi
zbierać i wysyłać dwa rodzaje danych:
\begin{itemize}
    \item dane telemetryczne,
    \item zdjęcia wykonane w czasie lotu
\end{itemize}

Dane telemetryczne muszą zawierać informacje o pozycji
drona oraz identyfikować maszynę za pomocą unikatowego
identyfikatora oraz numeru lotu. Przesyłany jest też
poziom naładowania baterii oraz informacja, czy w obecnym
czasie prowadzone jest nagrywanie.

\section{Protokoły wymiany danych}

\subsection{Dane telemetryczne}
Protokół do wymiany danych telemetrycznych powinien
wysyłać dane w postaci binarnej, gdyż jest to bardziej
efektywne niż kodowanie ich w postaci tekstowej (na
przykład w formacie \texttt{JSON}).

Protokół powinien być też w łatwy sposób rozszerzalny,
pozwalając w przyszłości zredefiniować część wysyłanych
pakietów lub dodać nowe dane, bez tracenia kompatybilności
wstecznej bądź konieczności przebudowania całego systemu.
Poszczególne komponenty systemu będą pisane w różnych
językach programowania - biorąc to pod uwagę, pożądaną
cechą protokołu jest też możliwość szybkiego przeportowania
go na inny język programowania. 

\subsection{Zdjęcia}
Efektywny przesył zdjęć oraz filmów to temat zbyt złożony
i wymagający, żeby poruszać go w treści pracy - system
powinien wykorzystywać dowolny prosty w implementacji
protokół wysyłania zdjęć. Architektura systemu powinna
zapewnić możliwość prostej wymiany tego komponentu, dzięki
czemu w przyszłości będzie możliwe zastąpienie go
przez bardziej zoptymalizowane rozwiązanie.

\section{Oprogramowanie serwerowe}

\subsection{Odbiór i multipleksowanie telemetrii}

\section{Oprogramowanie klienckie}


% Wielowirnikowce podłączone do systemu, muszą być
% wyposażone w kontroler lotu - umożliwiający
% autonomiczny lot, stabilizację oraz obsługujący 
% peryferia takie jak czujniki oraz silniki.

% Poza kontrolerem lotu, który zawiera jedynie oprogramowanie 
% niezbędnie do sterowania lotem i udostępniania
% strumienia telemetrii (zazwyczaj przez port szeregowy),
% na maszynie musi znaleźć się też drugi komputer - 
% do zastosowań bardziej ogólnych.