\chapter{Wymagania funkcjonalne systemu}

\section{Oprogramowanie na dronie}

Wielowirnikowce podłączone do systemu muszą być
zdolne do autonomicznego lotu -- w kontekście pracy
oznacza to zdolność do automatycznego startu,
lądowania, stabilizacji oraz samodzielnego
lotu do koordynatów GPS. W trakcie lotu, maszyna musi
zbierać i wysyłać dwa rodzaje danych:
\begin{itemize}
    \item dane telemetryczne,
    \item zdjęcia wykonane w czasie lotu.
\end{itemize}

Dane telemetryczne muszą zawierać informacje o pozycji
drona oraz identyfikować maszynę za pomocą unikatowego
identyfikatora oraz numeru lotu. Przesyłany jest też
poziom naładowania baterii oraz informacja, czy w obecnym
czasie prowadzone jest nagrywanie.

\section{Protokoły wymiany danych}

W przypadku systemu działającego autonomicznie, wymiana danych jest kluczowym
elementem pozwalającym na sprawdzanie poprawności działania
i diagnozowania błędów w systemie. Podczas lotów testowych
często nie ma możliwości bezpośredniej obserwacji systemu
lub ingerencji w jego sposób działania. Odpowiednia architektura
zbierająca i archiwizująca dane z lotów pozwala znacznie szybciej 
wykryć potencjalne problemy i zapobiec krytycznym błędom. 

\subsection{Dane telemetryczne}
Protokół do wymiany danych telemetrycznych powinien
wysyłać dane w postaci binarnej, gdyż jest to bardziej
efektywne niż kodowanie ich w postaci tekstowej (na
przykład w formacie \texttt{JSON}, typowym dla języków
wysokopoziomowych - wykorzystywanym w technologiach webowych).

Protokół powinien być w łatwy sposób rozszerzalny,
pozwalając w przyszłości zredefiniować część wysyłanych
pakietów lub dodać nowe dane, bez tracenia kompatybilności
wstecznej bądź konieczności przebudowania całego systemu.
Poszczególne komponenty systemu będą pisane w różnych
językach programowania -- biorąc to pod uwagę, pożądaną
cechą protokołu jest też możliwość szybkiego przeportowania
go na inny język programowania. 

\subsection{Zdjęcia wykonane w trakcie lotu}
Efektywny przesył zdjęć oraz filmów to temat zbyt obszerny
i wymagający, żeby poruszać go w treści pracy - system
powinien wykorzystywać dowolny prosty w implementacji
protokół wysyłania zdjęć. Architektura systemu powinna
zapewnić możliwość prostej wymiany tego komponentu, dzięki
czemu w przyszłości będzie możliwe zastąpienie go
przez bardziej zoptymalizowane rozwiązanie.

\section{Oprogramowanie serwerowe}

Serwer webowy jest komponentem, który odbiera, archiwizuje i przekazuje
dane nadchodzące z dronów do aplikacji klienckiej. Jest punktem centralnym systemu,
wykorzystywanym bezpośrednio przez wszystkie pozostałe elementy.

\subsection{Odbiór i multipleksowanie telemetrii}

Aby umożliwić diagnozowanie stanu systemu w czaie rzeczywistym -- szczególnie
stanu wykonujących lot wielowirnikowców, telemetria nadchodząca z maszyn nie może być
jedynie archiwizowana na dysku serwera centralnego. Konieczną funkcjonalnością jest
przesyłanie jej w czasie rzeczywistym do wielu jednocześnie podłączonych klientów.

Umożliwi to podjęcie akcji w przypadku wykrycia krytycznego błędu, który mógłby
zakończyć się uszkodzeniem bądź rozbiciem drona, ułatwi też wykonywanie testów - zarówno
na rzeczywistych maszynach, jak i wykorzystujących symulatory lotu.

\section{Oprogramowanie klienckie}

Aplikacja kliencka skupiona jest wokół trzech funkcjonalności:

\begin{enumerate}
    \item planowanie tras i harmonogramu lotów,
    \item odbiór telemetrii i zdjęć z dronów w czasie rzeczywistym,
    \item przegląd i analiza telemetrii oraz zdjęć zarchiwizowanych z poprzednich lotów.
\end{enumerate}

Kluczowym elementem aplikacji klienckiej jest obsługa mapy -- wszystkie
wymienione funkcjonalności wymagają wizualizacji nadchodzących danych geograficznych,
rozszerzonych o dodatkowe informacje (na przykład godzina przelotu przez dany punkt lub
zarejestrowane w danym miejscu zdjęcia i wykryte na nich obiekty).