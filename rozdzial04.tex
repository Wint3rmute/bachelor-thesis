\chapter{Wdrażanie systemu}

% Napisać jeszcze raz, wtf
Jak zaznaczono w zakresie pracy (\ref{intro_scope}), celem pracy jest
budowa systemu o architekturze, która pozwoli na automatyzację wdrożeń.
W tym rozdziale opisane są technologie i praktyki, które zostały 
zastosowane, aby zautomatyzować proces wdrażania systemu oraz proces
dodawania do systemu nowych funkcjonalności.

\section{Konteneryzacja}

Wszystkie elementy systemu docelowo są uruchamiane w kontenerach.
Wykorzystanym systemem konteneryzacji jest \texttt{docker}\cite{docker}.
% można dopisać w rozdziale03 że jest Dockerfile w każdym repo
Konteneryzacja ułatwia proces przenoszenia oprogramowania 
z maszyny na maszynę. Pozwala też na pełne zautomatyzowanie
procesu budowania oprogramowania i instalacji koniecznych
składników środowiska uruchomieniowego. Konteneryzacja
aplikacji wymusza na programiście napisanie skryptu, w wyniku
którego w kontenerze zostanie zainstalowane wszystkie konieczne 
oprogramowanie.

\subsection{Zachowywanie stanu systemu plików w kontenerze}

Ważną różnicą, jaka odróżnia kontenery od aplikacji zainstalowanych
w sposób konwencjonalny, jest \textbf{bezstanowość}. Jeśli komputer,
na którym działają kontenery zostanie zresetowany, bądź kontener zostanie
usunięty i utworzony na nowo (na przykład w celu aktualizacji), system
plików w kontenerze zostanie zresetowany do stanu, w jakim znajdował się 
w momencie uruchomienia (takim, jaki jest zapisany w \textit{obrazie kontenera}).

\texttt{Docker} pozwala rozwiązać ten problem za pomocą mechanizmu zwanego 
woluminami (\textit{docker volumes})\cite{docker_volumes}.
Za pomocą woluminów, dowolny fragment systemu
plików kontenera może zostać zmapowany do nieulotnej pamięci komputera, na którym 
działa kontener. W ten sposób zapewnia się persystencję plików baz danych, czy innych
plików, które mają być przetrzymywane przez aplikację.\\ % newline CAN be removed

\textit{Uwaga:} nie wszystkie kontenery potrzebują zachowywać pliki - kontener
odpowiedzialny za hostowanie statycznej strony internetowej nie potrzebuje
zapamiętywać zmian w plikach strony. 

\section{Automatyczne budowanie komponentów systemu}

Jak wspomniano w rozdziale \ref{repo_structure}, poświęconym strukturze repozytoriów,
wszystkie repozytoria zawierają plik \texttt{.gitlab-ci.yml}. Plik ten definiuje skrypty,
jakie wykona system \texttt{CI/CD}, gdy do repozytorium zostanie dodany nowy kod.

Ponieważ konteneryzacja pozwala na zupełne zautomatyzowanie procesu budowania,
skrypty w systemie \texttt{CI/CD} są w stanie samodzielnie zbudować kontener,
zawierający konkretny komponent systemu. Następnie, kontener wysyłany jest
do \textit{rejestru kontenerów} -- specjalnego repozytorium, w którym można
przechowywać i wersjonować kontenery. Skrypt systemu CI/CD automatycznie oznacza
kontener za pomocą nazwy brancha, na której został zbudowany. Kontenery pochodzące
z brancha \textit{master}, uruchamiane są na serwerze produkcyjnym.

\begin{figure}[H]
	\centering\small
	\caption{
      Proces automatycznej budowy kontenerów 
	}
	\label{telemetry_backend_architecture}
\tikzstyle{arrow} = [thick,->,>=stealth]

\begin{tikzpicture}[node distance=4cm,auto,>=stealth']
    
    \node[] (programmer) {Programista};
    \node[right of = programmer] (repo) {Repozytorium};
    \node[right of = repo] (ci_cd) {System \texttt{CI/CD}};
    \node[right of = ci_cd] (registry) {Rejestr kontenerów};

    \node[below of = programmer, node distance=6cm] (programmer_ground) {};
    \node[below of = repo, node distance=6cm] (repo_ground) {};
    \node[below of = ci_cd, node distance=6cm] (ci_cd_ground) {};
    \node[below of = registry, node distance=6cm] (registry_ground) {};
    %
    \draw (repo) -- (repo_ground);
    \draw (programmer) -- (programmer_ground);
    \draw (ci_cd) -- (ci_cd_ground);
    \draw (registry) -- (registry_ground);


    \draw[->] ($(programmer)!0.20!(programmer_ground)$) -- node[above,scale=1,midway]{Nowy \textit{commit}} ($(repo)!0.20!(repo_ground)$);
    % \draw[<-] ($(programmer)!0.35!(programmer_ground)$) -- node[above,scale=1,midway]{Text} ($(repo)!0.35!(repo_ground)$);
    \draw[->,align=center] ($(repo)!0.35!(repo_ground)$) -- node[above,scale=1,midway]{Zainicjowanie procesu\\budowania} ($(ci_cd)!0.35!(ci_cd_ground)$);

    \draw[->,align=center] ($(ci_cd)!0.60!(ci_cd_ground)$) -- node[above,scale=1,midway]{ Wysłanie zbudowanego \\ i przetestowanego \\ konteneru do rejestru} ($(registry)!0.60!(registry_ground)$);
    
    \draw[<-,align=center] ($(repo)!0.70!(repo_ground)$) -- node[above,scale=1,midway]{Status i logi \\z systemu \texttt{CI/CD}} ($(ci_cd)!0.70!(ci_cd_ground)$);
    \draw[<-,align=center] ($(programmer)!0.90!(programmer_ground)$) -- node[above,scale=1,midway]{Informacja, czy \\ \textit{commit} przeszedł testy} ($(repo)!0.90!(repo_ground)$);
\end{tikzpicture}

\end{figure}

\section{Automatyczne aktualizacje kontenerów}

Pobranie i uruchomienie kontenera, zawierającego dany komponent systemu
nie wymaga przechodzenia przez proces instalacji czy konfiguracji - wystarczy
pobrać obraz kontenera z rejestru i uruchomić go:

\begin{lstlisting}[language=bash, label=list:docker_clone_run_example,caption={Pobranie i uruchomienie obrazu dockera, zawierającego aplikację}, basicstyle=\footnotesize\ttfamily]
# Zapisane jako zmienna, aby poprawić czytelność przykładu 
IMAGE=registry.gitlab.com/academic-aviation-club/gavron/frontend:master

docker pull $IMAGE
docker run \
    -d \ # detach: kontener działa w tle, nie wysyła logów do terminala
    -p 5000:5000 \ # mapuje port 5000 z kontenera do portu na komputerze  
    --name frontend \ # Nadaje nazwę uruchomionemu kontenerowi
    $IMAGE # Nazwa obrazu do uruchomienia 
\end{lstlisting}

W przypadku, gdy na serwerze działa już kontener z aplikacją, należy
pobrać nowy obraz kontenera, usunąc obecnie działający kontener i uruchomić
nowy obraz:

\begin{lstlisting}[language=bash, label=list:docker_update_container,caption={Aktualizacja kontenerów}, basicstyle=\footnotesize\ttfamily]
IMAGE=registry.gitlab.com/academic-aviation-club/gavron/frontend:master

docker pull $IMAGE # Pobranie najnowszej wersji kontenera

# Nazwa kontenera nadana w poprzednim przykładzie, za pomocą --name
docker stop frontend

# Usunięcie kontenera w starej wersji (jego stanu, zmian w systemie plików)
docker rm frontend

# Tak samo jak w przykładzie powyżej
docker run \
    -d \ # detach: kontener działa w tle, nie wysyła logów do terminala
    -p 5000:5000 \ # mapuje port 5000 z kontenera do portu na komputerze  
    --name frontend \ # Nadaje nazwę uruchomionemu kontenerowi
    $CONTAINER # Nazwa kontenera do uruchomienia 
\end{lstlisting}

W przypadku wielu równolegle działających kontenerów, ręczna aktualizacja 
każdej działającej aplikacji jest zadaniem niepotrzebnie czasochłonnym.
Wprowadza też dodatkowy punkt, w którym może zajść pomyłka -- na przykład
pominięcie jednego z kontenerów przy aktualizacji.

Projekt wykorzystuje narzędzie \texttt{Ouroboros} do automatycznej aktualizacji
uruchomionych na serwerze kontenerów. \texttt{Ouroboros} w regularnych odstępach
czasu sprawdza, czy uruchomione na serwerze kontenery nie wymagają aktualizacji.
Jeśli w rejestrze dostępna jest nowa wersja obrazu kontenera, pobiera ją i zastępuje
nią obecnie działający kontener.

\section{Orkiestracja systemu złożonego z wielu kontenerów - \texttt{docker-compose}}