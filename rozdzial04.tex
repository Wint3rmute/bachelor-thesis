\chapter{Redakcja pracy}
\section{Układ pracy}
Standardowo praca powinna być zredagowana w następującym układzie:

\noindent\fbox{\begin{minipage}{\dimexpr\textwidth-2\fboxsep-2\fboxrule\relax}
\begin{quote}
\item Strona tytułowa
\item Strona z dedykacją (opcjonalna)
\item Spis treści  
\item Spis rysunków (opcjonalny)
\item Spis tabel (opcjonalny)
\item Skróty (wykaz opcjonalny)
\item 1. Wstęp 
\begin{quote}
\item 1.1 Cel i zakres pracy 
\item 1.2 Układ pracy 
\end{quote}
\item 2. Kolejny rozdział
\begin{quote}
\item 2.1 Sekcja
\begin{quote}
\item 2.1.1 Podsekcja
\begin{quote}
\item Nienumerowana podpodsekcja
\begin{quote}
\item Paragraf
\end{quote}
\end{quote}
\end{quote}
\end{quote}
\item $\ldots$
\item \#. Podsumownie i wnioski
\item Literatura
\item A. Dodatek
\begin{quote}
\item A.1 Sekcja w dodatku
\end{quote}
\item $\ldots$
\item \$. Zawartość płyty CD/DVD
\item Indeks rzeczowy (opcjonalny)
\end{quote}
\end{minipage}}\\

Spis treści -- powinien być generowany automatycznie, z podaniem tytułów i numerów stron. Typ czcionki oraz wielkość liter spisu treści powinny być takie same jak w niniejszym wzorcu.

Spis rysunków, Spis tabel -- powinny być generowane automatycznie (podobnie jak Spis treści). Elementy te są opcjonalne (robienie osobnego spisu, w którym na przykład są tylko dwie pozycje specjalnie nie ma sensu).

Wstęp -- pierwszy rozdział, w którym powinien znaleźć się opis dziedziny, w jakiej osadzona jest praca, oraz wyjaśnienie motywacji do podjęcia tematu.  
W sekcji ,,Cel i zakres'' powinien znaleźć się opis celu oraz zadań do wykonania, zaś w sekcji ,,Układ pracy'' -- opis zawartości kolejnych rozdziałów.

Podsumowanie -- w rozdziale tym powinny być zamieszczone: podsumowanie uzyskanych efektów oraz wnioski końcowe wynikające z realizacji celu pracy dyplomowej.

Literatura -- wykaz źródeł wykorzystanych w pracy (do każdego źródła musi istnieć odpowiednie cytowanie w tekście). Wykaz ten powinien być generowany automatycznie.

Dodatki -- miejsce na zamieszczanie informacji dodatkowych, jak: Instrukcja wdrożeniowa, Instrukcja uruchomieniowa, Podręcznik użytkownika itp.
Osobny dodatek powinien być przeznaczony na opis zawartości dołączonej płyty CD/DVD. Założono, że będzie to zawsze ostatni dodatek.

Indeks rzeczowy -- miejsce na zamieszczenie kluczowych wyrazów, do których czytelnik będzie chciał sięgnąć. Indeks powinien być generowany automatycznie. Jego załączanie jest opcjonalne.
\section{Styl}
\label{sec:Styl}
Zasady pisania pracy (przy okazji można tu zaobserwować efekt wyrównania wpisów występujących na liście wyliczeniowej uzależnione od długości etykiety):
\begin{enumerate}[labelwidth=\widthof{\ref{last-item}},label=\arabic*.]
\item Praca dyplomowa powinna być napisana w  formie bezosobowej (,,w pracy pokazano ...''). Taki styl przyjęto na uczelniach w naszym kraju, choć w krajach anglosaskich preferuje się redagowanie treści w pierwszej osobie.
\item W tekście pracy można odwołać się do myśli autora, ale nie w pierwszej osobie, tylko poprzez wyrażenia typu: ,,autor wykazał, że ...''. 
\item Odwołując się do rysunków i tabel należy używać zwrotów typu: ,,na rysunku pokazano ...'', ,,w tabeli zamieszczono ...'' (tabela i rysunek to twory nieżywotne, więc ,,rysunek pokazuje'' jest niepoprawnym zwrotem).
\item Praca powinna być napisana językiem formalnym, bez wyrażeń żargonowych (,,sejwowanie'' i ,,downloadowanie''), nieformalnych czy zbyt ozdobnych (,,najznamienitszym przykładem tego niebywałego postępu ...'')
\item Pisząc pracę należy dbać o poprawność stylistyczną wypowiedzi
\begin{itemize}
\item trzeba pamiętać, do czego stosuje się ,,liczba'', a do czego ,,ilość'',
\item nie ,,szereg funkcji'' tylko ,,wiele funkcji'',
\item redagowane zdania nie powinny być zbyt długie (lepiej podzielić zdanie wielokrotnie złożone na pojedyncze zdania),
\item itp.
\end{itemize}
\item Zawartość rozdziałów powinna być dobrze wyważona. Nie wolno więc generować sekcji i podsekcji, które mają zbyt mało tekstu lub znacząco różnią się objętością. Zbyt krótkie podrozdziały można zaobserwować w przykładowym rozdziale~\ref{chap:podsumowanie}.
\item Niedopuszczalne jest pozostawienie w pracy błędów ortograficznych czy tzw.\ literówek -- można je przecież znaleźć i skorygować
automatycznie. \addtocounter{enumi}{9997} 
\item  Niedopuszczalne jest pozostawienie w pracy błędów ortograficznych czy tzw.\ literówek -- można je przecież znaleźć i skorygować
automatycznie. \label{last-item}
\end{enumerate}


